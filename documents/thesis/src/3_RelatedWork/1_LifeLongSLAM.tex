\subsection{Lifelong SLAM}

The term Lifelong SLAM has varied definitions throughout the literature, but common themes of robustness to dynamic change, and long-term operation can be seen. For simplicity, this research adopts the definition provided by Shi et al. \cite{shiAreWeReady2020}, describing lifelong SLAM as the ability for a robot to generate, maintain, and localize within a map of a particular environment over an extended period of time. This is in contradistinction to the standard SLAM operating model, which tends to require short timeframes due to the assumption of a static environment. Unlike standard SLAM, lifelong SLAM is expected to operate despite environmental changes such as moved objects, lighting changes, dynamic objects within sensor view, etc. Some topics which fall under the umbrella of lifelong SLAM field have already been discussed, such map reuse. An exploration of previous lifelong SLAM work is provided below, grouped by the specific challenge addressed by the research.

\subsubsection{Place Recognition}

There is an expectation that the robot may get lost at some point during long-term operations. In these situations, the ability to recognize previously visited areas allows the system to determine the robot's current pose within the existing map, allowing the system to continue using an existing map, and keeping the same coordinate frame. Place recognition in standard SLAM is not trivial, requiring the ability to recognize previously mapped areas by visual or geometric similarity to sensor data. In lifelong SLAM, the problem becomes more challenging, as relocalization methods must handle the possibility of visual or geometric changes to the environment.

ORB-SLAM3 performs place recognition through the use of a visual bag-of-words approach, producing a hierarchical database of localized frames based on the visual features observed within the frame \cite{camposORBSLAM3AccurateOpenSource2021}\cite{galvez-lopezBagsBinaryWords2012}. This database is queried using the visual features observed by the camera, with candidate positions identified by visual similarity. However, as implemented, ORB-SLAM3's place recognition method does not align with the goals of lifelong SLAM, as it makes no effort to handle dynamic objects or 

\subsubsection{Map Reuse}
\subsubsection{Map Maintenance}
\subsubsection{Change Detection}
\subsubsection{Dynamic Object Detection}
