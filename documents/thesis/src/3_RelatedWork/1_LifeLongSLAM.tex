\subsection{Lifelong SLAM}

The term Lifelong SLAM has varied definitions throughout the literature, but common themes of robustness to dynamic change, and long-term operation can be seen. For simplicity, this research adopts the definition provided by Shi et al. \cite{shiAreWeReady2020}, describing lifelong SLAM as the ability for a robot to generate, maintain, and localize within a map of a particular environment over an extended period of time. This is in contradistinction to the standard SLAM operating model, which tends to require short timeframes due to the assumption of a static environment. Unlike standard SLAM, lifelong SLAM is expected to operate despite environmental changes such as moved objects, lighting changes, dynamic objects within sensor view, etc. This research falls under the topic of map maintenance, but has downstream effects on other operations related to lifelong SLAM.

\subsubsection{Map Maintenance}

KV-SLAM relies on the ability to identify feature correspondences between new images, and the existing map to perform all basic functions. Therefore, to operate effectively, the map should reflect the current state of the environment as closely as possible. Lifelong SLAM carries the requirement that the system operates despite environmental changes, motivating the need for methods which can update the map to match or X the current environment. ORB-SLAM3 does not perform map maintenance in this sense, instead opting to continuously add new observations to the map \cite{camposORBSLAM3AccurateOpenSource2021}. This is effective in static environments, as few new points will be added when operating in previously visited areas. However, any changes to the environment will result in new map features superimposed over old, resulting in a map that does not represent the current environment effectively, and which grows continuously as it observes dynamic changes.

Implementations of map maintenance take several forms, with the common goal being the identification and removal of discrepancies between the map and the environment.