\subsection{Lifelong SLAM}

Lifelong SLAM refers to the fie