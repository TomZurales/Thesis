\subsection{Lifelong SLAM}

The term Lifelong SLAM has varied definitions throughout the literature, but common themes of robustness to dynamic change, and long-term operation can be seen. For simplicity, this research adopts the definition provided by Shi et al. \cite{shiAreWeReady2020}, describing lifelong SLAM as the ability for a robot to generate, maintain, and localize within a map of a particular environment over an extended period of time. This is in contradistinction to the standard SLAM operating model, which tends to require short timeframes due to the assumption of a static environment. Unlike standard SLAM, lifelong SLAM is expected to operate despite environmental changes such as moved objects, lighting changes, dynamic objects within sensor view, etc. This research falls under the topic of map maintenance, but has downstream effects on other operations related to lifelong SLAM.

\subsubsection{Map Maintenance}

In SLAM, map maintenance refers to the various operations which improve the map's performance for future operations. These operations generally into two categories: optimization, and discrepancy correction. Map optimization refers to operations aimed at reduce global error in the map, such as ORB-SLAM3's global bundle adjustment operation \cite{camposORBSLAM3AccurateOpenSource2021}, which attempts to minimize the global reprojection error across the entire map by adjusting keyframe and map point positions within known sensor noise parameters. Additionally, optimization can identify and remove redundant data, reducing overall map size and complexity. Kurz et al. 