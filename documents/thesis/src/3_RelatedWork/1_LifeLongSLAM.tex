\subsection{Lifelong SLAM}

The term Lifelong SLAM has varied definitions throughout the literature, but common themes of robustness to dynamic change, and long-term operation can be seen. For simplicity, this research adopts the definition provided by Shi et al. \cite{shiAreWeReady2020}, describing lifelong SLAM as the ability for a robot to generate, maintain, and localize within a map of a particular environment over an extended period of time. This is in contradistinction to the standard SLAM operating model, which tends to require short timeframes due to the assumption of a static environment. Unlike standard SLAM, lifelong SLAM is expected to operate despite environmental changes such as moved objects, lighting changes, dynamic objects within sensor view, etc. This research falls under the topic of map maintenance, but has downstream effects on other operations related to lifelong SLAM.

\subsubsection{Map Maintenance}

KV-SLAM relies on the ability to identify feature correspondences between new images, and the existing map to perform all basic functions. Therefore, to operate effectively, the map should reflect the current state of the environment as closely as possible. Lifelong SLAM carries the requirement 