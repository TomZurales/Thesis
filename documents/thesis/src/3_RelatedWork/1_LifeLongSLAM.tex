\subsection{Lifelong SLAM}

The term Lifelong SLAM has varied definitions throughout the literature, but a common theme of robustness, and long-term operation can be seen. For simplicity, this research adopts the definition provided by Shi et al. \cite{shiAreWeReady2020}, describing lifelong SLAM as the ability for a robot to generate, maintain, and localize within a map of a particular environment over an extended period of time. This is in contradistinction to the standard SLAM operating model, which tends to require short timeframes due to the assumption of a static environment. Unlike standard SLAM, lifelong SLAM operates despite environmental changes such as moved objects, lighting changes, dynamic objects within sensor view, etc. Some topics related to lifelong SLAM have already been discussed, such as place recognition, and map reuse. An exploration of previous work is provided below, grouped by the specific challenge addressed by the research.

\subsubsection{Change Detection}
