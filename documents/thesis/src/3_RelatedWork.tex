\section{Related Work}
\label{sec:related_work}

This chapter contextualizes this thesis within the broader body of SLAM research. The topics discussed are selected for their relevance to two previously outlined objectives: the identification and removal of data that could cause instability, and the improvement of SLAM performance over long time horizons. While the goals of this research align with the topics of semantic rejection and lifelong SLAM, the methods applied are more closely related to change detection, point stability and re-observability confidence. The following sections explore the key ideas from these topics, and relate them to the proposed research.

Various 

\subsection{Change Detection}

Change detection in SLAM refers to identifying discrepencies between the current sensor observations, and the existing map. It supports lifelong SLAM and episodic operations by allowing maps to identify and remove outdated elements.

The implementation of change detection depends on sensor modality and map representation, but can be split into three categories: appearance-based, geometry-based, and semantic change detection.

\subsubsection{Appearance-Based Change Detection}

Appearance-based methods identify differences in the visual features between the map and image-based observations. 

The implementations underlying these operations vary with sensor modality and map representation.




\subparagraph{Object Level}
Larsen et al. \cite{larsenChangeDetectionModel} utilized pairs of images to identify geometric inconsistencies in existing 3D maps, allowing for the identification of individual added or removed objects.

\subparagraph{Geometric Level}
Derner et al. \cite{dernerChangeDetectionUsing2021} assign a weight to each point to quantify its stability to movement.

\subsection{Lifelong SLAM}

Lifelong SLAM has been used to refer to several topics related to SLAM research, but the 

\subsection{Change Detection}

Change detection is the 
