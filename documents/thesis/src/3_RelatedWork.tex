\section{Related Work}
\label{sec:related_work}

This chapter contextualizes this thesis within the broader body of SLAM research. The topics discussed are selected for their relevance to two previously outlined objectives: the identification and removal of data that could cause instability, and the improvement of SLAM performance over long time horizons. While the goals of this research align with the topics of semantic rejection and lifelong SLAM, the methods applied are more closely related to change detection, point stability and re-observability confidence. The following sections explore the key ideas from these topics, and relate them to the proposed research.

\subsection{Change Detection}

Change detection refers to the identification of differences between a previous and a current observation. This function has wide applicability, but in the context of SLAM, change detection refers to the identification of discrepancies between the map and the environment. In lifelong SLAM operations, change detection is often used to assist with map maintenance tasks by identifying and updating previously mapped areas with the latest observations. This requires the ability to:
\begin{enumerate}
    \item Correlate new observations to their respective position on the map
    \item Update the map in a meaningful way as to reflect the changes
\end{enumerate}
The implementations underlying these requirements varies greatly with sensor modality and map representation.

\subparagraph{Pointcloud-Pointcloud Registration}



\subparagraph{Object Level}
Larsen et al. \cite{larsenChangeDetectionModel} utilized pairs of images to identify geometric inconsistencies in existing 3D maps, allowing for the identification of individual added or removed objects.

\subparagraph{Geometric Level}
Derner et al. \cite{dernerChangeDetectionUsing2021} assign a weight to each point to quantify its stability to movement.

\subsection{Lifelong SLAM}

The term Lifelong SLAM has varied definitions throughout the literature, but common themes of robustness to dynamic change, and long-term operation can be seen. For simplicity, this research adopts the definition provided by Shi et al. \cite{shiAreWeReady2020}, describing lifelong SLAM as the ability for a robot to generate, maintain, and localize within a map of a particular environment over an extended period of time. This is in contradistinction to the standard SLAM operating model, which tends to require short timeframes due to the assumption of a static environment. Unlike standard SLAM, lifelong SLAM is expected to operate despite environmental changes such as moved objects, lighting changes, dynamic objects within sensor view, etc. Some topics which fall under the umbrella of lifelong SLAM field have already been discussed, such map reuse. An exploration of previous lifelong SLAM work is provided below, grouped by the specific challenge addressed by the research.

\subsubsection{Place Recognition}

As previously discussed, the ability to recognize previously visited areas allows a SLAM system to perform important operations like relocalization and loop closure, increasing robustness and reducing global map error. Place recognition in standard SLAM is already nontrivial, requiring methods to efficiently identify similarity between historical data and current sensor readings. In lifelong SLAM, the problem becomes more challenging, as relocalization methods must handle the possibility of visual or geometric changes to the environment.

ORB-SLAM3 performs place recognition with a visual bag-of-words approach, producing a hierarchical database of localized frames based on the visual features observed within the frame \cite{camposORBSLAM3AccurateOpenSource2021}\cite{galvez-lopezBagsBinaryWords2012}. This database is queried with visual features observed by the camera, providing candidate positions identified by visual similarity. However, as implemented, ORB-SLAM3's place recognition does not fulfil the objectives of lifelong SLAM. While the system may successfully place recognize despite some dynamic changes, no effort is made to recognize these changes or update internal representations based on observations, leading to failures in dynamic situations.



\subsubsection{Map Reuse}
\subsubsection{Map Maintenance}
\subsubsection{Change Detection}
\subsubsection{Dynamic Object Detection}


\subsection{Change Detection}

Change detection refers to the process of detecting discrepencies between  the detection and identification of 
