\section{Related Work}
\label{sec:related_work}

This chapter contextualizes this thesis within the broader body of SLAM research. The topics discussed are selected for their relevance to two previously outlined objectives: the identification and removal of data that could cause instability, and the improvement of SLAM performance over long time horizons. While the goals of this research align with the topics of semantic rejection and lifelong SLAM, the methods applied are more closely related to change detection, point stability and re-observability confidence. The following sections explore the key ideas from these topics, and relate them to the proposed research.

\subsection{Change Detection}

In computer vision, change detection refers to the identification of differenceIn the context of SLAM, change detection refers to the identification of discrepancies between the map and the environment. In lifelong SLAM operations, change detection is often used to assist with map maintenance tasks by identifying areas of change and updating them to match the current environment. However, change detection has been used for many other purposes. Change detection methods can be split based on the abstraction level at which they operate.

\subparagraph{Object Level}
Larsen et al. \cite{larsenChangeDetectionModel} utilized pairs of images to identify geometric inconsistencies in existing 3D maps, allowing for the identification of individual added or removed objects.

\subparagraph{Geometric Level}
Derner et al. \cite{dernerChangeDetectionUsing2021} assign a weight to each point to quantify its stability to movement.

\subsection{Lifelong SLAM}

Lifelong SLAM has been used to refer to several topics related to SLAM research, but the 

\subsection{Change Detection}

Change detection is the 
