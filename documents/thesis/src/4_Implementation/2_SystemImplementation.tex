\subsection{Observability Model Implementation}

The implementations explored in this thesis serve as representative examples of various methods for modeling map point observability. While not an exhaustive exploration of all possible methods or implementations, these examples provide a foundation for future research.

Each model described below is initialized with a vector of inputs representing various tunable parameters for the model. The final selection of these inputs will be tuned through an optimization process described in Section \ref{sec:parameter_tuning}. Each model provides functions for querying the internal representation's estimate of $P(S^{\boldsymbol{v}}|E)$ for a given viewpoint $\boldsymbol{v}$, integrating new observations into the internal representation, and optionally, receiving feedback from the Existence Probability Estimator.

\paragraph{K-Nearest Neighbors Model}

A simple method for modeling the observability is the use of a K-nearest-neighbors clustering method. In this model, the estimation is the number of positive observations in the k-nearest observations divided by k. The replacement strategy utilizes feedback from the existence estimator, replacing the nearest observation with the new observation if $\Delta P(E) > \tau$ where $\tau$ is a threshold found through optimization.

\begin{algorithm}[H]
    \caption{Query Viewpoint Probability}
    \label{alg:query}
    \begin{algorithmic}[1]
        \Require Viewpoint $v = (\theta, \phi)$
        \REQUIRE Global list of observations $(\theta_i, \phi_i, \text{is\_positive}_i)$
        \REQUIRE Global parameters $k$, $\texttt{max\_angle}$
        \ENSURE Probability of existence based on nearest neighbors

        \STATE $candidates \gets [\;]$
        \FORALL{$obs \in observations$}
        \STATE $obs\_view \gets (\theta_i, \phi_i)$
        \IF{$\texttt{angular\_dist}(v, obs\_view) \leq \texttt{max\_angle}$}
        \STATE $dist \gets \texttt{euclidean\_dist}(v, obs\_view)$
        \STATE $candidates.\texttt{append}((dist, \text{is\_positive}_i))$
        \ENDIF
        \ENDFOR
        \STATE Sort $candidates$ by $dist$ in ascending order
        \STATE $k\_neighbors \gets$ first $\min(k, \texttt{length}(candidates))$ entries of $candidates$
        \IF{$\texttt{length}(k\_neighbors) = 0$}
        \RETURN $0.5$
        \ENDIF
        \STATE $positives \gets$ count of entries in $k\_neighbors$ where $\text{is\_positive} = \text{True}$
        \RETURN $positives / \texttt{length}(k\_neighbors)$
    \end{algorithmic}
\end{algorithm}

Parameters to Optimize: $[n, k, \tau]$

\paragraph{Binned Model}

In this model, each observation is assigned to a bin based on the unit direction of observation. Each bin stores the maximum distance from which it was observed, total number of observations within that distance, and the number of positive observations within that distance. The estimate is the number of positive observations within the distance divided by the total number of observations, or 0 if the observation occurs outside the observability shell.

\paragraph{Continuous Model}

\subsubsection{Icosahedral Shell Implementation}
\label{sec:icos_construction}

\subsection{Bayesian Update Implementation}

\subsection{Existence Estimation Framework Implementation}