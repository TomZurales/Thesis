\subsection{System Implementation}

The implementations explored in this thesis serve as representative examples of possible methods for modeling . While not exhaustive, these methods provide a foundation for future research into map point observability modeling.

\paragraph{K-Nearest Neighbors Model}

A simple method for modeling the observability is the use of a K-nearest-neighbors clustering method. In this model, the estimation is the number of positive observations in the k-nearest observations divided by k. The replacement strategy utilizes feedback from the existence estimator, replacing the nearest observation with the new observation if $\Delta P(E) > \tau$ where $\tau$ is a threshold found through optimization.

Parameters to Optimize: $[n, k, \tau]$

\paragraph{Binned Model}

In this model, each observation is assigned to a bin based on the unit direction of observation. Each bin stores the maximum distance from which it was observed, total number of observations within that distance, and the number of positive observations within that distance. The estimate is the number of positive observations within the distance divided by the total number of observations, or 0 if the observation occurs outside the observability shell.

\paragraph{Continuous Model}

\subsubsection{Icosahedral Shell Implementation}
\label{sec:icos_construction}