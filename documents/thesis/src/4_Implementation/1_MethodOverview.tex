\subsection{Method Overview}

Let's step back for a moment recall the goal of this research. We are attempting to identify map points which, while previously viewed, are no longer observable due to environmental changes. To that end, we assign a probability of existence value to each map point which we update over time. When a point is observed from a particular viewpoint, the confidence in that point's existence increases. However, if we fail to observe a map point from a viewpoint where it \it have been visible

\subsubsection{Viewpoint-Aware Observability of Map Points}

% define what we mean by viewpoint aware observability; a mapping between observation unit direction vector and seen/not seen binary observations

Viewpoint-aware observability is a function characterized by a unit direction vector $\mathbf{v}\in\mathbb{S}^2$ and a distance $d\in\mathbb{R}^+$
$$
    \mathbb{S}^2\rightarrow\{0,1\}
$$

% Justify why distance needs to be accounted for, use a diagram to show that by including distance, we can account for occlusions in the model

This model of observability does not account for occlusions. Therefore, we instead utilize the function
$$
    \mathbb{S}
$$

% Say what we are trying to build; a model which can produce a probability that a point will be seen given an observation direction and a distance


\subsubsection{Observability Shell Representations}

\subsubsection{Existence Probability and Update Rule}

\subsubsection{Application to Point Pruning and Selection}
