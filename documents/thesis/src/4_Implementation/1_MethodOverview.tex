\subsection{Method Overview}

This section details our method for achieving the goals laid out in Section \ref{objectives}. The primary objective is to identify map points which, while previously viewed, are no longer visible due to environmental changes. To address this, we assign an incrementally updated probability of existence value to each map point. Observations of a map point increase our overall confidence in its existence. Conversely, failure to observe a map point from a viewpoint where it \textit{should} have been visible lowers our confidence in its existence. Determining whether a map point should be observable from a given viewpoint is not trivial, motivating the need for a viewpoint-aware observability model. The purpose of the model is to integrate historical observability data to estimate observability across all possible viewpoints.

\subsubsection{Viewpoint-Aware Observability of Map Points}

% define what we mean by viewpoint aware observability; a mapping between observation unit direction vector and seen/not seen binary observations

We define viewpoint aware observability as a function
\begin{align*}
    \boldsymbol{f:}\mathbb{S}^2\times\mathbb{R}^+\rightarrow\{0,1\}
\end{align*}
where $\mathbf{v}\in\mathbb{S}^2$ is a unit direction vector pointing from the map point to the observer, and $d\in\mathbb{R}^+$ is the Euclidean distance between the point and observer. The function $\boldsymbol{f}(\mathbf{v}, d)\rightarrow\{0,1\}$ returns 1 if the map point is considered visible from that viewpoint, and 0 otherwise.

% Justify why distance needs to be accounted for, use a diagram to show that by including distance, we can account for occlusions in the model

The inclusion of distance in the model is necessary to handle occlusions in the environment. Figure \ref{fig:enhanced_general_slam_pipeline} shows a situation. By integrating distance data into the model, we can accurately determine whether an observation warrants modifying the point's probability of existence, or if it should be considered occluded.

% Say what we are trying to build; a model which can produce a probability that a point will be seen given an observation direction and a distance

The model must perform two functions: updates and queries. Updates modify the model's internal representation of a map point's observability, while queries ask \"What is the probability that I see the map point from this viewpoint?\". The goal of the model is to provide answers to queries which are as accurate as possible based on the updates received up until that point. With perfect knowledge of the map point's observability across the domain of viewpoints, every query response would be a 0 or a 1, however, this would require infinite observations to produce. Instead, we implement the model probabilistically, allowing a finite number of observations to estimate the point's observability across the domain, and returning a probability of observation in the range [0, 1]. Figure \ref{fig:2d_observability} provides a simplified 2D representation of a map point $\boldsymbol{p}$'s ideal observability.

\begin{figure}[htbp]
    \centering
    % TODO: Add actual figure content
    \caption{A simplified 2D representation of a map point's ideal observability pattern. The point is observable from certain viewing directions (shown in green) but occluded from others (shown in red).}
    \label{fig:2d_observability}
\end{figure}

\subsubsection{Observability Shell Representations}

We model the viewpoint-aware observability of each map point using a geometric shell representation. This shell structure encodes the directional observation history of the point and is incrementally updated as new observations are processed. The shell can be implemented using various geometric constructs, and in both continuous and discrete modalities. This work proposes two observability shell representations: a discrete version implemented on an icosahedral shell, and a continuous version implemented on a spherical shell. For both structures, update and query functions are provided.

The discrete icosahedral shell allows the full observation history to be integrated into a set of discrete bins, requiring constant additional space per map point. On the other hand, the continuous model stores data for each observation, meaning the space requirements increase with time. This growth can be bounded by limiting the number of stored observations, for example using a fixed size queue.

Below we introduce the mathematical formulation for both the discrete and continuous shells. For each modality, the mathematics behind the update and query functions are defined and discussed.

\paragraph{Discrete Icosahedral Shell}

The icosahedron was selected for the discrete shell because it contains the highest face count of the convex Platonic solids. This provides 20 bins across the domain of viewpoints, each with identical coverage of the domain.

\subparagraph{Icosahedral Shell Construction}

For the purposes of the mathematical derivation of the update and query functions, the construction of the Icosahedral shell is arbitrary. The specific construction used in this work will be discussed in Section \ref{sec:icos_construction}, but for this high level overview, we assume the existence of the following constructs:

\begin{itemize}
    \item A vector of faces $\mathbf{F}$ of length 20
    \item A vector of face centers $\mathbf{F}_center$
\end{itemize}

\subsubsection{Existence Probability and Update Rule}

\subsubsection{Application to Point Pruning and Selection}
