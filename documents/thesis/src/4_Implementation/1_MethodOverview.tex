\subsection{Method Overview}

This section details our method for achieving the goals laid out in Section \ref{objectives}. The primary objective is to identify map points which, while previously viewed, are no longer visible due to environmental changes. To address this, we assign an incrementally updated probability of existence value to each map point. Observations of a map point increase our overall confidence in its existence. Conversely, failure to observe a map point from a viewpoint where it \textit{should} have been visible lowers our confidence in its existence. Determining whether a map point should be observable from a given viewpoint is not trivial, motivating the need for a viewpoint-aware observability model.

\subsubsection{Viewpoint-Aware Observability of Map Points}

% define what we mean by viewpoint aware observability; a mapping between observation unit direction vector and seen/not seen binary observations

Viewpoint-aware observability is a function characterized by a unit direction vector $\mathbf{v}\in\mathbb{S}^2$ and a distance $d\in\mathbb{R}^+$, producing a binary observable/not observable output.
\begin{align*}
    \mathbb{S}^2\times\mathbb{R}^+\rightarrow\{0,1\} \\
    f(sdf)
\end{align*}

% Justify why distance needs to be accounted for, use a diagram to show that by including distance, we can account for occlusions in the model

This model of observability does not account for occlusions. Therefore, we instead utilize the function
$$
    \mathbb{S}
$$

% Say what we are trying to build; a model which can produce a probability that a point will be seen given an observation direction and a distance


\subsubsection{Observability Shell Representations}

\subsubsection{Existence Probability and Update Rule}

\subsubsection{Application to Point Pruning and Selection}
