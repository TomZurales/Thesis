\section{Introduction}

This section provides an introduction to the limitations of SLAM systems, and discusses how the research prevented in this thesis intends to solve those shortcomings.

\subsection{Motivation}

% Describe the project that inspired this work; astrobee map reuse on the ISS.

The inspiration for this research came from a project with the goal of augmenting the Astrobee robot navigation system on the International Space Station (ISS) with keypoint-based visual SLAM. At the time of the project, Astrobee navigation consisted of localization on a pre-generated map, which required both astronaut and ground team time to produce, leading to infrequent map updates \cite{soussanAstroLocEfficientRobust2022}. By augmenting the Astrobee with SLAM, we aimed to reduce the need for manual map updates and improve the robot's autonomous capabilities. Additionally, we planned to reuse maps between Astrobee operation sessions to produce consistent localization estimates, registered to the established coordinate frame of the ISS. However, we found that the ISS presents unique challenges for robot navigation. The structure of the station is stable, but the contents and position of equipment changes frequently. Consequently, maps created aboard the ISS quickly become obsolete. Attempting to load outdated maps for reuse degraded Astrobee's navigation performance, limiting the robot's ability to perform autonomously and complete its mission objectives \cite{zuralesCollaborativeSensingMapping2024}.

% Describe why the robotics world would benefit from SLAM systems that can operate over long periods of time in changing environments.

This failure of autonomy for long-term, multi-session operations in changing environments presented a problem for Astrobee, just as it does for many other robotics applications. If solved to the point where SLAM systems could operate robustly, over long periods of time, and in highly dynamic environments, integration of robotics into human environments would be greatly simplified.

\subsection{Objectives and Scope}

% Define the objectives of this research. What are the overall goals. What should be different once this is implemented

% Describe the scope of the research; what limits are placed on the objectives by this scope?


This research 

This research intends to build upon the previously developed probability models, in order to distill the update step of each map point's probability of existence into a simple Bayesian update step. The goals for this model are as follows:

1. To utilize constant additional space for each map point
2. To complete the update step in constant time
3. To resist updating confidence levels with redundant data

\subsection{Contribution}

Through this research, we introduce an incrementally updated directional confidence model for the existence of map points. This model differs from other point removal optimizations in several ways. First, this implementation avoids the use of neural networks, facilitating use on resource constrained hardware without facilities optimized to run them. Second, while other probability based point removal optimizations have been developed, this model introduces the idea of utilizing a continuously updated perspective dependent shell of metadata for each keypoint, which can be used to reduce the problem of point existence to a simple Bayesian update step. This implementation allows perspective of observation to play a role in the point's existence probability update step, and avoids some of the common a priori work such as prior estimation common to other point removal optimization techniques. This shell is implemented in both finite and continuous modalities, utilizing regular convex polyhedral shells in the finite case, and von-meiser fisher distributions on the sphere in the continuous case.

To facilitate future research, this model is released as an open-source library, which is compatible with any keypoint based visual SLAM implementation. Additionally, a collection of co-registered visual-inertial and LIDAR datasets is provided, containing instances of multiple traversals through the same environments with changes to scene contents. Information regarding the locations of these environmental changes is included in the dataset, facilitating the benchmarking of point removal optimization implementations.

\subsection{Road Map}

Chapter X of this thesis discusses the background of the general SLAM problem, covering the history, use cases, and general pipeline utilized by SLAM systems. This is followed by a deeper dive into keypoint-based visual SLAM, the sensor modality targeted by this research. We provide a brief survey of widely utilized extensions to the core SLAM algorithm which target deficiencies in the core pipeline. Finally, we discuss fields outside the scope of SLAM which provide insight and methodology into this research.

Chapter X discusses works related to this research, specifically focusing on extensions targeting improved performance in dynamic situations, with additional focus given to those methods which utilize point removal optimizations.