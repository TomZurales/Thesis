\section{Introduction}

% Describe the entire thesis in a few sentences. What is the problem, solution and results?
The research presented in this thesis addresses a challenge common in long-term usage of SLAM systems: the need to continuously localize and map in non-static environments. We propose a novel, lightweight method for culling outdated map points from previously generated maps using a probabilistic, viewpoint-aware model of map point observability. This method is designed to operate efficiently on mobile robotics platforms with limited computational resources. We demonstrate the effectiveness of this approach in identifying outdated map points, improving localization accuracy, and improving the robustness of SLAM between different runs of the system.

\subsection{Problem Context}

% Talk about time on the astrobee project and the problems encountered

The International Space Station (ISS) has three robotic passengers on board. These robots, known as Astrobees, were launched with the intention of being a platform for zero-gravity robotics research, a camera rig for recording astronaut activity, and a sensor platform for performing station keeping and monitoring tasks \cite{smithASTROBEENEWPLATFORM}. Core to each of these missions is the ability for Astrobee to autonomously localize and navigate throughout the U.S. Orbital Segment (USOS). Unfortunately, the harsh realities of navigation on board the ISS quickly became apparent. The high-level structure of the station is constant, but the visual features within the station are not stable, changing due to equipment resupplies, hardware movements, astronaut activity, and lighting changes. The irregularity in the visual features of the ISS degraded the performance of the Astrobee localizer, which relied on visual features stored in a pre-generated map \cite{soussanAstroLocEfficientRobust2022}. Mapping the ISS required input from ground crew and astronauts, which, due to the tight timetables associated with ISS operations, led to infrequent map updates. These outdated maps caused frequent lost states, inaccurate localization estimations, and overall poor performance of the navigation stack. Eventually, Astrobee activities were restricted almost entirely to the Japanese Experiment Module (JEM), a relatively small area which could be kept visually stable, and could be quickly remapped when needed \cite{carlinoLessonsLearnedAstrobee}.

A project was proposed to augment Astrobee's navigation stack with a Keypoint-Based Visual SLAM (KV-SLAM) system. The idea behind this change was to have the SLAM system load an initial map from the previous mission, localize within this map, then update the map with the latest visual features from the environment. At the end of the operation, the updated map could be saved off for use in the next mission \cite{zuralesCollaborativeSensingMapping2024}. For this research, we will call this operational concept of iteratively updating previous maps between operations \textit{episodic SLAM}. Testing of the SLAM system showed presented the same issues which were encountered by the Astrobee localizer when attempting to reuse outdated maps. The conjecture stemming from these results was that for reused maps to be useful, they must not only be updated with the latest visual features, but they must also remove features which have become outdated.

Three key lessons were learned from the Astrobee work which motivated the research presented here. First, the episodic operating model has potential in a wide variety of applications, providing a method for regular, large scale environmental monitoring of changing environments such as factories, warehouses and offices. Second, map reuse is a powerful tool for robotic systems operating in the same environment over long periods of time, allowing for the preservation of structural map data and the global reference frame. Finally, SLAM systems are not inherently robust to outdated maps, showing degraded performance as the initial map diverges from the true environment.


\subsection{Problem Statement}

% Describe what causes SLAM to struggle in long-term, multi-session operations in changing environments

The concept of multi-session operations involves a SLAM system which only operates intermittently. The primary use case for this class of system would be for care taking operations in environments which are only sometimes inhabited by people, Eg. factories during business hours. The benefits of this operating method is that the SLAM system can avoid operating during highly dynamic times, with the tradeoff being that the environment may be significantly changed between sessions. Map reuse allows the system to avoid starting from zero at the beginning of every session.

% Describe the problems with existing map point culling methods, and why they are bad for low-compute platforms
Numerous methods for identifying and filtering non-static data from SLAM systems have been implemented to great success, allowing SLAM to perform well in highly dynamic environments. These implementations will be discussed in detail in Section \ref{sec:related_work}. A common theme amongst many of the best performing systems is the utilization of a learning model for identifying dynamic objects in an image frame, and ignoring any image features detected on them. This is effective, as it generates a map consisting entirely of what should be static objects, unlikely to change or move. The downside of this methodology is that it requires compute resources capable of running image segmentation and semantic identification in real time and in parallel with the other SLAM processes. Other methods of dynamic data removal utilize probability modeling, treating each image feature as an element which has some probability of existing. These methods have also seen success in improving operations in dynamic environments, but are also limited. Of the implementations which will be discussed in Section \ref{related_work}, not all maintain an understanding of the directions from which a point is observable, and none maintain a full history of a points directional observability. This can lead to points being culled from the map while they are temporarily obscured, or maintained long after they should have been culled.

These limitations are particularly problematic for Mobile Autonomous Vehicles (MAVs), where compute resources are constrained by weight, power and size requirements. In scenarios where a compute limited MAV operates in a dynamic environment, the current set of available methods for dynamic data elimination pose either untenable computation demands, or fail to take full advantage of directionality of observation when performing map pruning. This motivates the development of a lightweight, probability based method for outdated map point culling which explicitly models each map points directional observability to better calculate map point existence.



\subsection{Objectives and Scope}
\label{objectives}

The objective of this research is to accurately identify outdated map points in reused keypoint-based visual SLAM maps beyond the current capability of existing change detection methods. 

\subsection{Research Questions}

\subsubsection*{Question 1: How can the historical viewpoint-dependent observability of points be modeled in order to meet the objectives laid out in Objective 1?}

To integrate historical viewpoint-dependent observability into existing change detection approaches requires a compact model of observability at the point level. Three model implementations are explored in Section \ref{sec:observability_models}, and evaluated in Section \ref{sec:observability_model_eval}.

\subsubsection*{Question 2: How does the integration of historical observability data affect the estimation of point existence confidence?}

Each of the three model implementations is integrated into an existing change detection model, discussed in Section \ref{sec:existence_confidence}, which utilizes a Bayesian update step to iteratively update a confidence estimation of the map point's existence. Each of the three models is evaluated for speed, accuracy and robustness in Section \ref{sec:existence_confidence_eval}.

\subsubsection*{Question 3: Can the change detection model be integrated into an existing KV-SLAM system in a way that meets the goals of Objective 2?}

The extended confidence update is integrated into ORB-SLAM3, a popular KV-SLAM system, for evaluation against a series of datasets intended to test the system across a range of simple and challenging situations. The integration is discussed in Section \ref{sec:orb_slam3_integration}, and the evaluation is covered in Section \ref{sec:orb_slam3_integration_eval}.

\subsection{Contribution}

Through this research, we contribute three things to the subject of long-term map reuse in KV-SLAM. First, we release an open-source library which provides reusable access to the mathematical implementation of the VAMPE model. Next, a modified version of ORB-SLAM3 with the VAMPE model integrated is released for ease of further testing and research. Finally, we provide a collection of datasets covering situations particularly challenging for intermittently run KV-SLAM systems for demonstration and comparison of the system's capabilities.

% The library: a lightweight, viewpoint-aware model of map point existence, designed to be utilized in keypoint-based visual SLAM systems.

The VAMPE library is a C++ implementation of the viewpoint-aware probability shell, with hooks for simple integration into any existing KV-SLAM implementation. The library

% The SLAM implementation: An implementation of the library in ORB-SLAM 3 to demonstrate its effectiveness.

% The datasets: A set of datasets and experimental results demonstrating how the library improves SLAM performance.

Through this research, we introduce an incrementally updated directional confidence model for the existence of map points. This model differs from other point removal optimizations in several ways. First, this implementation avoids the use of neural networks, facilitating use on resource constrained hardware without facilities optimized to run them. Second, while other probability based point removal optimizations have been developed, this model introduces the idea of utilizing a continuously updated perspective dependent shell of metadata for each keypoint, which can be used to reduce the problem of point existence to a simple Bayesian update step. This implementation allows perspective of observation to play a role in the point's existence probability update step, and avoids some of the common a priori work such as prior estimation common to other point removal optimization techniques. This shell is implemented in both finite and continuous modalities, utilizing regular convex polyhedral shells in the finite case, and von-meiser fisher distributions on the sphere in the continuous case.

To facilitate future research, this model is released as an open-source library, which is compatible with any keypoint based visual SLAM implementation. Additionally, a collection of co-registered visual-inertial and LIDAR datasets is provided, containing instances of multiple traversals through the same environments with changes to scene contents. Information regarding the locations of these environmental changes is included in the dataset, facilitating the benchmarking of point removal optimization implementations.

\subsection{Road Map}

Chapter \ref{sec:background}
Chapter \ref{sec:background}
Chapter \ref{sec:background}