\section{Introduction}

% Describe the entire thesis in a few sentences. What is the problem, solution and results?
The research presented in this thesis addresses a challenge common in long-term usage of SLAM systems: the need to continuously localize and map in non-static environments. We propose a novel, lightweight method for culling outdated map points from previously generated maps using a probabilistic, viewpoint-aware model of map point observability. This method is designed to operate efficiently on mobile robotics platforms with limited computational resources. We demonstrate the effectiveness of this approach in identifying outdated map points, improving localization accuracy, and improving the robustness of SLAM between different runs of the system.

\subsection{Motivation}

The inspiration for this research came from a project to utilize keypoint based visual SLAM on the Astrobee robots on the International Space Station (ISS). At the time of the project, Astrobee navigation was accomplished through localization on a pre-generated map, which required both astronaut and ground team intervention to produce \cite{soussanAstroLocEfficientRobust2022}. Due to the tight schedules and high costs associated with ISS astronaut work, this map creation happened very rarely.

The ISS is a challenging place to perform robot navigation tasks. The primary structure of the station remains the same, with nodes always connecting to the same nodes, but regular resupply, hardware replacement, temporary experimental setups, and general crew living can cause the station to gradually change from day to day. This high variation in the internal features of the space station caused the Astrobee maps to quickly go out of date, severely limiting the robot's autonomous capabilities and utilities as a research platform \cite{PLACEHOLDER}.

It was the intention of the project to run keypoint based visual SLAM on board the Astrobees, allowing them the capabilities to create and update their own maps and eliminating the need for astronaut intervention. In order to be useful for ISS operations, the SLAM algorithm had to output its map and localization estimates in the established ISS coordinate frame \cite{PLACEHOLDER}. To accomplish this, the decision was made early on to build off of a SLAM system which included the ability to load and merge previously generated maps. Assuming the system localizes in the pregenerated map, the coordinate frame can be preserved between mapping runs by merging the new data into the old map. This is where the non-static nature of the ISS produces problems for SLAM. If a map is generated a significant time before it is used, there is ample opportunity for the visual features of the environment to change, leading to poor performance when attempting to localize within this previously generated map. While out of scope for the previous project, the researchers noted that these out of date visual features in the previously generated map posed a challenge for SLAM operations in similarly dynamic environments.

\subsubsection{Problem Statement}

A primary assumption in SLAM is that the environment remains static throughout operations. It is obvious 

\subsubsection{Research Questions}

Does a probability model which identifies and culls outdated map points provide significant improvements to relocalization and tracking during map reuse in keypoint-based SLAM?
To what extent do dynamic changes in a map affect mapping and relocalization performance?
Can this be used as a heuristic to determine when to re-enable mapping on MAVs?

The spacial understanding provided by SLAM is not only useful, but necessary for systems intending to operate in and interact with physical environments. Virtual reality, robotics, and industrial automation all make use of SLAM to generate an internal map of the local and global environments. SLAM does have the distinction of being a "solved" problem in the ideal case. If an agent is able to perfectly measure the environment, and is guaranteed to make correct data associations, then a perfect map can be generated and the agent's location within that map can be known with certainty. This ideal case makes several assumptions, paramount of which is the existence of ideal sensors, but there is a secondary assumption that the state of the world does not change.

It is obvious that the assumption of a static, unchanging world does not hold in practice. In fact, the inspiration for this research came from attempts to perform localization on the International Space Station (ISS). As of the time of publication, there are three mobile autonomous vehicles (MAVs) on board the ISS known as Astrobees. While used for numerous experiments and product development tasks on the ISS, the Astrobees are prone to failure due to loss of localization. The primary cause of this localization failure is the constant changes occurring on the ISS, including equipment changes, resupply missions, and any other activities which may change the visual features of the ISS.

The situation on the ISS is not unique, and would be experienced by any agent running SLAM in all but the most tightly controlled environments. VR goggles using use SLAM to determine their position in 3D space within a room must contend with new objects which are placed in the room, the changing images shown on the TV, people walking in and out of view, etc. Robots operating in an office environment must be robust to moved desks, the movement of people, and more. Even robotic operations in an unmanned space station (a situation proposed for the Lunar Gateway project) would need to be able to perform despite changing lighting conditions, moved equipment, other MAVs in view, etc. Overall, a requirement that a SLAM agent gets to operate in a pristine, unchanging environment would be an insurmountable barrier for real world use.

The field of research into making SLAM perform over long timeframes has been called lifetime SLAM[], eluding to the fact that SLAM systems with the requirement for an unchanging environment will still be able to operate successfully over short timeframes, but will struggle with missions which take place over multiple days, weeks, months, or years.

This research is specific to a subset of the greater SLAM problem known as keypoint-based, visual SLAM. The distinctions between these will be discussed in detail in the Background section, but a high-level overview is offered here. Keypoint-based visual SLAM operates on images taken over time. The core principal involves constructing a sparse 3D map of image features which are identifiable from multiple perspectives. This is accomplished through photogrammetry methods such as the 5-point method, which allows the depths of 5 matched pairs of points, and the parallax transformation to be determined from two 2d images [FACT CHECK and CITATION].

<!-- Go into a few more details about some of the other fields of research which are used by SLAM -->

There are numerous keypoint-based visual SLAM implementations seeing use today, but all follow a relatively straightforward pipeline, defined as follows:

1. Determine an initial set of 3D points from two images with sufficient parallax
2. For subsequent images, match features with previously identified 3D points
3. Determine the transformation between the previous image and the new image which maximizes the number of map point alignments

Implementation differences tend to come from optimization steps, pruning of redundant data, anomaly handling, and additional sensor integrations. In order to achieve lifetime SLAM, the system must be able to determine what data is remaining static, what data is changing, and act accordingly. Imagine an art gallery with many paintings on the walls. If you were to visit this gallery on two separate occasions one year apart, the paintings on the walls may change, but you are able to identify that you are in the same gallery. People perform this contextual elimination of data which may change on a subconscious level []. Replicating this contextual awareness in SLAM allows systems to recognize and focus on unchanging data while ignoring dynamic data which could clutter and confuse the agent's internal map.

The benefits of eliminating data which is not helpful for long-term operations are plentiful. Just like culling redundant data, culling dynamic data reduces the overall size of the map. This reduces storage capacity requirements, while providing a smaller pool of data through which processes like Random Sample Consensus (RanSaC) need to search. A keypoint which was seen on an object which is later moved will always be an outlier in subsequent observations. Through culling of these dynamic data points, we can improve the speed and robustness of the several optimization steps, reduce overall system hardware requirements, and increase confidence in the accuracy and validity of the produced maps.

Numerous methods for improving SLAM's performance over long timeframes have been implemented, pushing the field of SLAM to the point where it is now seeing deployment in numerous dynamic environments. A discussion of several of these implementations takes place in the Background section, with a focus on each implementation's strengths, weaknesses, and overall effectiveness. An area that remains lacking is implementations on MAVs with limited compute. Due to their mobile nature, MAVs are inherently compute limited, as any additional weight and power consumption decreases capability and operation time. This prevents the inclusion of many popular models for dynamic data elimination such as image segmentation and semantic identification. Other methods utilizing statistical methods for point existence exist, but fail to fully utilize the vast array of data to update their predictions

\subsection{Problem Statement}

% A potential future use case, and 
A future use case of stochastic KV-SLAM would be for operations in environments which are intermittently occupied, e.g. factories, warehouses, office buildings, etc. The benefit of this method is that the robot can avoid operating during highly dynamic times, with the trade-off being that the environment may be significantly changed between sessions of operation. In the case of Astrobee, the navigation system was incapable of dealing with these inconsistencies between the internal map and the real world, regularly requiring astronaut assistance to recover. Trust in autonomous systems is already precarious, quickly falling upon the observation of failure \cite{robinetteEffectRobotPerformance2017}. Robotic SLAM failures can lead to navigational instability, task failure, and even the possibility of physical damage to the surrounding environment or the robot itself\cite{nahavandiComprehensiveReviewAutonomous2025a}. Therefore, in order for autonomous SLAM based robotic navigation to see widespread use, it must be accurate in its estimations, and capable of withstanding a wide variety of failure cases.

This research focuses on reducing failures caused by the existence of outdated map features in reused maps during stochastic KV-SLAM operations. There are three critical issues which appear in systems which do not handle outdated map points sufficiently well. The first issue relates to the size of the map, as only adding newly observed features without removing outdated ones results in a monotonically increasing map size. This puts an ever-increasing strain on compute resources to deal with the large map, and increases the complexity of optimization steps. The second issue is in regard to relocalization, the process by which a KV-SLAM system redetermines its position within the map after being lost. The process of relocalization starts with identifying candidate positions within the map which are visually similar to the current camera image. A high proportion of outdated map points interferes with this candidate selection, as the map's understanding of the visual features of the environment no longer match reality. The final issue affects the RANSAC process, a function which is run regularly as part of the process of determining the 3D transformation between successive frames. The RANSAC algorithm selects a random sample of points from the map and the new image to be localized, and attempts to determine a consistent 3D transformation which explains the correspondence. High proportions of outdated map points increase the risks of the RANSAC algorithm failing to find a consistent pose for the new frame, even if one exists. The map representation, relocalization, and RANSAC will be covered in greater detail in Section \ref{sec:kv_slam_background}.

There are two ways to prevent the failure modes specific to outdated map points in reused maps. The first option is to proactively prevent map features which have potential to become outdated (i.e. they are located on an object that can move) from ever entering the map. The second method is to retroactively identify and remove outdated features from the map once it is loaded. The mechanism at play in the proactive case tends to be semantic identification and segmentation of the input images, allowing the semantic labels to determine whether map points which fall on certain objects are kept or not. These semantic processes require significant compute resources, often making them unsuitable for compute constrained applications. In the reactive case, the methods stray away from the field of semantics, and towards the field of change detection. These methods work by identifying inconsistencies between the map and the camera observations, flagging points for removal once confidence in their existence drops below some threshold. However, a gap can be noticed in the change detection methods available today. Both the semantic and change-detection methods will be discussed in greater detail in Section \ref{sec:related_work}.

To this end, numerous methods for identifying and filtering non-static data from SLAM systems have been implemented to great success, allowing SLAM to perform well in highly dynamic environments. These implementations will be discussed in detail in Section \ref{sec:related_work}. A common theme amongst many of the best performing systems is the utilization of a learning model for identifying dynamic objects in an image frame, and ignoring any image features detected on them. This is effective, as it generates a map consisting entirely of what should be static objects, unlikely to change or move. The downside of this methodology is that it requires compute resources capable of running image segmentation and semantic identification in real time and in parallel with the other SLAM processes. Other methods of dynamic data removal utilize probability modeling, treating each image feature as an element which has some probability of existing. These methods have also seen success in improving operations in dynamic environments, but are also limited. Of the implementations which will be discussed in Section \ref{related_work}, not all maintain an understanding of the directions from which a point is observable, and none maintain a full history of a points directional observability. This can lead to points being culled from the map while they are temporarily obscured, or maintained long after they should have been culled.

% Describe why the robotics world would benefit from SLAM systems that can operate over long periods of time in changing environments.

Despite the challenges and risks associated with failure, research into improving SLAM's capabilities to long time horizons and more dynamic environments is extremely active. This is because of the significant benefits afforded by SLAM to operate in arbitrary environments, generate high quality maps, and produce accurate localization estimations. These capabilities are critical for integrating robotic autonomy into everyday life. In the specific domain of long-term, multi-session operations in intermittently static environments such as warehouses, factories, and even the ISS, improvements to SLAM's ability to localize and map move the technology closer to widespread adoption.  The work presented here intends to push the field slightly towards that goal by improving localization within reused maps by identifying and eliminating outdated map points.

% Describe what causes SLAM to struggle in long-term, multi-session operations in changing environments

The concept of multi-session operations involves a SLAM system which only operates intermittently. The primary use case for this class of system would be for care taking operations in environments which are only sometimes inhabited by people, E.g. factories during business hours.

% Describe the problems with existing map point culling methods, and why they are bad for low-compute platforms



\subsection{Research Questions}

This research introduces a novel method of map point removal for reused maps in keypoint-based visual SLAM systems. This method utilizes a model of feature observability across the domain of possible viewpoints, and 

\subsubsection*{Question 1. }

% Is a viewpoint-aware model of map point existence an effective way to detect and cull outdated map points?

1. Is a viewpoint-aware model of map point existence an effective way to detect and cull outdated map points from pre-generated SLAM maps?

Successfully answering this question will require implementation of the keypoint elimination model, and benchmarking the time it takes to cull known outdated map points, comparing against other keypoint elimination metrics from the literature.

% What are the effects of this model on SLAM performance?

2. When integrated into an existing SLAM system, does the model perform keypoint culling without significant detrimental effects on the system's performance metrics?

The answer to this question requires comparing both the unmodified and modified SLAM outputs against ground truth trajectories. This requires the creation and utilization of datasets designed to challenge existing SLAM systems while exercising the need for keypoint elimination from reused maps.

% Can this be implemented in a lightweight way capable of running on low-compute power platforms?

3. Can the model be integrated into an existing SLAM system in such a way that it does not significantly increase compute requirements to run the system?

This is implementation specific, but the answer to this question can be found by benchmarking the completion times for key functions in both the modified and unmodified systems.

\subsection{Objectives and Scope}
\label{objectives}

% Define the scope of this work, focusing exclusively on map reuse between runs of SLAM in semi-static environments.

This research is intended to improve keypoint-based visual SLAM performance in situations where a pre-generated map is loaded to lock in a known coordinate frame, and in changing environments. The key distinction from other operation modes is the repeated operations in the same environment, with an expectation of the maintenance of large scale static features, but without the expectation for smaller visual features to remain the same between runs. This is accomplished through the introduction of a viewpoint-aware probabilistic model for the existence of map points. This research studies the effects of removing map points from pre-generated maps using this model, and the downstream effects on optimization, relocalization and map accuracy.

% Define the objectives of the model; to be lightweight and effective at identifying outdated map points, and to be usable in real-time SLAM systems.

The objective of the model is to produce an accurate method of representing the global observability of each map point in a map. The model should be lightweight, requiring only small amounts of additional space per map point. The model should integrate all observational data into its understanding of the map point's observability, while rejecting redundant data. Identification of map points which are outdated should be fast and accurate. Map points which remain visible should be robustly maintained.

% Define the objectives of the implementation, not to hurt performance in static environments, and to improve performance with previously generated maps in semi-static environments.

This model will be integrated into ORB-SLAM3 for testing purposes, ensuring we meet the implementation objectives of speed, accuracy and improvement. The model should not introduce significant increases in compute requirements. Additionally, performance in static environments should remain the same regardless of model implementation. However, under the specific scenerio of map reuse in a changing environment, performance (as meausred by trajectiory accuracy to ground truth) should be improved.

\subsection{Contribution}

Through this research, we contribute three things to the subject of long-term map reuse in KV-SLAM. First, we release an open-source library which provides reusable access to the mathematical implementation of the VAMPE model. Next, a modified version of ORB-SLAM3 with the VAMPE model integrated is released for ease of further testing and research. Finally, we provide a collection of datasets covering situations particularly challenging for intermittently run KV-SLAM systems for demonstration and comparison of the system's capabilities.

% The library: a lightweight, viewpoint-aware model of map point existence, designed to be utilized in keypoint-based visual SLAM systems.
\subsubsection{The VAMPE Library}
The VAMPE library is a C++ implementation of the viewpoint-aware probability shell, with hooks for simple integration into any existing KV-SLAM implementation. Implementations for both a discrete viewpoint shell (implemented as an icosahedron) and a continuous viewpoint shell (implemented on a sphere) are made available.

% The SLAM implementation: An implementation of the library in ORB-SLAM 3 to demonstrate its effectiveness.
\subsubsection{ORB-SLAM3_VAMPE}
ORB-SLAM3, an extremely popular KV-SLAM implementation, is modified to utilize the VAMPE library for estimating the probability of map point existence within the global map. These probabilistic estimations are then used in the selection of keypoints for culling operations, regularly removing points which drop below a certain threshold, and for improving relocalization operations through an updated RANSAC implementation which weights high probability map points above low.

% The datasets: A set of datasets and experimental results demonstrating how the library improves SLAM performance.

The datasets provided are recorded using an OAK-D stereo camera, rigidly mounted with a Unitree L2 LIDAR. The LIDAR is used to produce a highly accurate ground-truth, co-registered to the camera's view. IMU data is provided for both the camera and LIDAR. A map is provided for each dataset, generated using an unmodified implementation of ORB-SLAM3, and co-registered to the ground truth. Bounding boxes within this frame are provided representing keypoints known to change between runs.

\subsection{Road Map}

Chapter \ref{sec:background} provides a high level overview of the concepts core to this research. Inner workings of KV-SLAM. Relocalization. RANSAC. Map Point Culling. Directional Statistics.

Chapter \ref{sec:related_work}
Chapter \ref{sec:implementation}
Chapter \ref{sec:analysis}
Chapter \ref{sec:conclusion}