\section{Background}
\label{background}

In this chapter, we provide a high level overview of the definitions, objectives, and history of SLAM, in addition to an overview of the common sensor modalities found in SLAM systems. Following this, we discuss the stages of the SLAM pipeline in the generic case, followed by a more in-depth exploration of keypoint-based visual SLAM, the sensor modality targeted by this research. Next, we look into several of the widely adopted extensions to the base SLAM pipeline which address core issues and enhance performance. A discussion of extensions which have similar goals to this research is held for the chapter on related works.

\subsection{SLAM Overview}

This research is acts as an extension to keypoint-based visual SLAM; a term which warrants some explanation. But before exploring the specifics of keypoint-based visual SLAM, some background on the general SLAM problem is required. The idea behind SLAM is to simultaneously produce a map of an environment, and determine the position of the observer within the map based on a set of sensor data. The process differs heavily based on the sensor types being utilized. For example, LIDAR provides a direct measurement of 3D distances from the sensor, while an RGB camera must calculate them from correspondences between multiple frames. While implementations differ heavily, a common SLAM pipeline could be described as follows:

\subsubsection{The SLAM Pipeline}

This stage is responsible for the creation of the initial map.

There have been hundreds of SLAM implementations for a wide variety of sensors, commonly targeting combinations of monocular, stereo or RGBD cameras, IMUs, LIDARs, etc.

Due to it providing the motivation for this project, the Astrobee robots will me mentioned several times throughout this work. The Astrobee project was motivated by the desire to research human/robot interaction, robotic automation and inspection, and to provide a research platform on which companies and researchers could deploy software and hardware for testing in a micro-gravity environment. The Astrobee platform has been used to develop satellite rendezvous control algorithms, grippers to capture tumbling orbital debris, inspection methods to autonomously detect anomalous operation, and many other space habitation focused endeavors.

\subsubsection{Keypoint-Based Visual SLAM}

The term Keypoint-Based Visual SLAM refers to the SLAM modality which primarily utilizes key points extracted from images as the primary means of mapping and navigating. This is distinct from systems like LIDAR-based SLAM, which utilize direct distance measurements from a LIDAR sensor, or Dense Visual SLAM, which

\subsubsection{Extensions to Core SLAM}

\subsection{Additional Fields of Research}

\subsubsection{Directional Probability}