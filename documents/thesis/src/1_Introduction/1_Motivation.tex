\subsection{Motivation}

% Describe the project that inspired this work; astrobee map reuse on the ISS.

The inspiration for this research came from a project with the goal of augmenting the Astrobee robot navigation system on the International Space Station (ISS) with keypoint-based visual SLAM. At the time of the project, Astrobee navigation consisted of localization on a pre-generated map, which required both astronaut and ground team time to produce, leading to infrequent map updates \cite{soussanAstroLocEfficientRobust2022}. By augmenting the Astrobee with SLAM, we aimed to reduce the need for manual map updates and improve the robot's autonomous capabilities. Additionally, we planned to reuse maps between Astrobee operation sessions to produce consistent localization estimates, registered to the established coordinate frame of the ISS. However, we found that the ISS presents unique challenges for robot navigation. The structure of the station is stable, but the contents and position of equipment changes frequently. Consequently, maps created aboard the ISS quickly become obsolete. Attempting to load outdated maps for reuse degraded Astrobee's navigation performance, limiting the robot's ability to perform autonomously and complete its mission objectives \cite{zuralesCollaborativeSensingMapping2024}.

% Describe why the robotics world would benefit from SLAM systems that can operate over long periods of time in changing environments.

This failure of autonomy for long-term, multi-session operations in changing environments presented a problem for Asrtobee, just as it does for many other robotics applications. If solved to the point where SLAM systems could operate robustly, over long periods of time, and in highly dynamic environments, integration of robotics into human environments would be greatly simplified.