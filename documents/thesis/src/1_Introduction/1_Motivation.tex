\subsection{Motivation}

The inspiration for this research came from a project to utilize keypoint based visual SLAM on the Astrobee robots on the International Space Station (ISS). At the time of the project, Astrobee navigation was accomplished through localization on a pre-generated map, which required both astronaut and ground team intervention to produce \cite{soussanAstroLocEfficientRobust2022}. Due to the tight schedules and high costs associated with ISS astronaut work, this map creation happened very rarely.

The ISS is a challenging place to perform robot navigation tasks. The primary structure of the station remains the same, with nodes always connecting to the same nodes, but regular resupply, hardware replacement, temporary experimental setups, and general crew living can cause the station to gradually change from day to day. This high variation in the internal features of the space station caused the Astrobee maps to quickly go out of date, severely limiting the robot's autonomous capabilities and utilities as a research platform \cite{PLACEHOLDER}.

It was the intention of the project to run keypoint based visual SLAM on board the Astrobees, allowing them the capabilities to create and update their own maps and eliminating the need for astronaut intervention. In order to be useful for ISS operations, the SLAM algorithm had to output its map and localization estimates in the established ISS coordinate frame \cite{PLACEHOLDER}. To accomplish this, the decision was made early on to build off of a SLAM system which included the ability to load and merge previously generated maps. Assuming the system localizes in the pregenerated map, the coordinate frame can be preserved between mapping runs by merging the new data into the old map. This is where the non-static nature of the ISS produces problems for SLAM. If a map is generated a significant time before it is used, there is ample opportunity for the visual features of the environment to change, leading to poor performance when attempting to localize within this previously generated map. While out of scope for the previous project, the researchers noted that these out of date visual features in the previously generated map posed a challenge for SLAM operations in similarly dynamic environments.

% Maybe bring up life long SLAM here?

\subsubsection{Problem Statement}

A primary assumption in the idealized SLAM problem is that the environment remains static throughout operations \cite{PLACEHOLDER}. This assumption often fails in real-world settings, and therefore must be accounted for and corrected. Numerous methods for identifying and filtering non-static data from SLAM systems have been implemented to great success, allowing SLAM to perform well in highly dynamic environments. These implementations will be discussed in detail in Section \ref{related_work}. A common theme amongst many of the best performing systems is the utilization of a learning model for identifying dynamic objects in an image frame, and ignoring any image features detected on them. This is effective, as it generates a map consisting entirely of what should be static objects, unlikely to change or move. The downside of this methodology is that it requires compute resources capable of running image segmentation and semantic identification in real time and in parallel with the other SLAM processes. Other methods of dynamic data removal utilize probability modeling, treating each image feature as an element which has some probability of existing. These methods have also seen success in improving operations in dynamic environments, but are also limited. Of the implementations which will be discussed in Section \ref{related_work}, not all maintain an understanding of the directions from which a point is observable, and none maintain a full history of a points directional observability. This can lead to points being culled from the map while they are temporarily obscured, or maintained long after they should have been culled.

These limitations are particularly problematic for Mobile Autonomous Vehicles (MAVs), where compute resources are constrained by weight, power and size requirements. In scenarios where a compute limited MAV operates in a dynamic environment, the current set of available methods for dynamic data elimination pose either untenable computation demands, or fail to take full advantage of directionality of observation when performing map pruning. This motivates the development of a lightweight, probability based method for outdated map point culling which explicitly models each map points directional observability to better calculate map point existence.

\subsubsection{Research Questions}

This thesis investigates whether a probabilistic model of map point existence utilizing directional data on map point observability can be used to identify and remove outdated 3D map points in keypoint-based visual SLAM systems. Specifically, this work seeks to answer the following research questions:

1. Can outdated map points be reliably identified using a directionally aware probability model where each map point maintains a record of its directional observability?

2. How does the inclusion of such a model impact the robustness and accuracy of a SLAM system's localization in environments exhibiting various levels of dynamic change?