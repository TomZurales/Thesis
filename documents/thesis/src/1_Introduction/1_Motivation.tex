\subsection{Motivation}

% Describe the project that inspired this work; astrobee map reuse on the ISS.

The inspiration for this research came from a project with the goal of augmenting the Astrobee robot navigation system on the International Space Station (ISS) with keypoint-based visual SLAM. At the time of the project, Astrobee navigation consisted of localization on a pre-generated map, which required both astronaut and ground team time to produce, leading to infrequent map updates \cite{soussanAstroLocEfficientRobust2022}. By augmenting the Astrobee with SLAM, we aimed to reduce the need for manual map updates and improve the robot's autonomous capabilities. Additionally, we planned to reuse maps between operations to produce consistent localization estimates, registered to the established coordinate frame of the ISS. However, we found that the ISS presents unique challenges for robot navigation. The structure of the station is stable, but the contents and position of equipment changes frequently. Consequently, maps created aboard the ISS quickly become obsolete. Attempting to load outdated maps for reuse degraded Astrobee's navigation performance, limiting the robot's ability to perform autonomously and complete its mission objectives \cite{zuralesCollaborativeSensingMapping2024}.

% Describe how issues with long-term SLAM prevent wide usage in semi-static environments; maps can be used for other tasks

Utilizing SLAM over long periods of time presents serious obstacles, preventing widespread usage. SLAM systems are typically designed to operate in static environments, where the map remains unchanged during the mapping process. However, in

% Describe what benefits we can gain from solving these problems; more robustness, better maps, more accurate localization, etc.

The goal central to the Astrobee project is to enable long-term usage of SLAM in semi-static environments (environments which may experience change while mapping is not occurring, but are relatively static during mapping operations) such as the ISS.