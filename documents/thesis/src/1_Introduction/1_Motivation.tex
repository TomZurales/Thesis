\subsection{Motivation}

The inspiration for this research came from a project to utilize keypoint based visual SLAM on the Astrobee robots on the International Space Station (ISS). At the time of the project, Astrobee navigation was accomplished through localization on a pre-generated map, which required both astronaut and ground team intervention to produce \cite{soussanAstroLocEfficientRobust2022}. Due to the tight schedules and high costs associated with ISS astronaut work, this map creation happened very rarely.

The ISS is a challenging place to perform robot navigation tasks. The primary structure of the station remains the same, with nodes always connecting to the same nodes, but regular resupply, hardware replacement, temporary experimental setups, and general crew living can cause the station to gradually change from day to day. This high variation in the internal features of the space station caused the Astrobee maps to quickly go out of date, severely limiting the robot's autonomous capabilities and utilities as a research platform \cite{PLACEHOLDER}.

It was the intention of the project to run keypoint based visual SLAM on board the Astrobees, allowing them the capabilities to create and update their own maps and eliminating the need for astronaut intervention. In order to be useful for ISS operations, the SLAM algorithm had to output its map and localization estimates in the established ISS coordinate frame \cite{PLACEHOLDER}. To accomplish this, the decision was made early on to build on a SLAM system which included the ability to load and merge previously generated maps. Assuming the system localizes in the pre-generated map, the coordinate frame can be preserved between mapping runs by merging the new data into the old map. This is where the non-static nature of the ISS produces problems for SLAM. If a map is generated a significant time before it is used, there is ample opportunity for the visual features of the environment to change, leading to poor performance when attempting to localize within this previously generated map. While out of scope for the previous project, the researchers noted that these out of date visual features in the previously generated map posed a challenge for SLAM operations in similarly dynamic environments.

