\subsection{Motivation}

% Describe the project that inspired this work; astrobee map reuse on the ISS.

The inspiration for this research came from a project with the goal of augmenting the Astrobee robot navigation system on the International Space Station (ISS) with keypoint-based visual SLAM. At the time of the project, Astrobee navigation consisted of localization on a pre-generated map, which required both astronaut and ground team time to produce, leading to infrequent map updates \cite{soussanAstroLocEfficientRobust2022}. By augmenting the Astrobee with SLAM, we aimed to reduce the need for manual map updates and improve the robot's autonomous capabilities. Additionally, we planned to reuse maps between operations to produce consistent localization estimates, registered to the established coordinate frame of the ISS. However, we found that the ISS presents unique challenges for robot navigation. The structure of the station is stable, but the contents and position of equipment changes frequently. Consequently, maps created aboard the ISS quickly become obsolete. Attempting to load outdated maps for reuse degraded Astrobee's navigation performance, limiting the robot's ability to perform autonomously and complete its mission objectives \cite{zuralesCollaborativeSensingMapping2024}.

% Describe how issues with long-term SLAM prevent wide usage in semi-static environments; maps can be used for other tasks

The requirements for a navigation system that can operate in semi-static environments over long periods of time are not unique to the ISS. Many environments such as factories, warehouses, homes, and hospitals can be considered semi-static, where the structure remains stable, but the visual features change significantly over time. Many small-scale deployments of robotic systems have taken place in semi-static environments, but have not yet reached the point of wide-spread adoption. In these environments, SLAM systems must be able to reuse maps in order to maintain accurate localization and mapping capabilities, while also adapting to the changes in the environment. The area of research with the goal of enabling such long-term operation of SLAM systems is known as \textit{Lifelong SLAM} \cite{bujancaRobustSLAMSystems2021}. The challenges of lifelong SLAM include the need to handle environmental changes, the ability to quickly and robustly relocalize after position estimation failures, the ability to manage large-scale maps and account for drift over time, and the ability to maintain a consistent coordinate frame for localization and mapping \cite{saputraVisualSLAMStructure2019}.

% Describe what benefits we can gain from solving these problems; more robustness, better maps, more accurate localization, etc.

The benefits of solving these problems are significant. The capability to perform accurate localization and mapping in uncontrolled, populated, and large scale environments would be a major step forward for the integration and deployment of autonomous robots in everyday life.