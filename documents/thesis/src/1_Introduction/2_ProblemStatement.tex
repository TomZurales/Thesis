\subsection{Problem Statement}

% A potential future use case, and 
A future use case of stochastic KV-SLAM would be for operations in environments which are intermittently occupied, e.g. factories, warehouses, office buildings, etc. The benefit of this method is that the robot can avoid operating during highly dynamic times, with the trade-off being that the environment may be significantly changed between sessions of operation. In the case of Astrobee, the navigation system was incapable of dealing with these inconsistencies between the internal map and the real world, regularly requiring astronaut assistance to recover. Trust in autonomous systems is already precarious, quickly falling upon the observation of failure \cite{robinetteEffectRobotPerformance2017}. Robotic SLAM failures can lead to navigational instability, task failure, and even the possibility of physical damage to the surrounding environment or the robot itself\cite{nahavandiComprehensiveReviewAutonomous2025a}. Therefore, in order for autonomous SLAM based robotic navigation to see widespread use, it must be accurate in its estimations, and capable of withstanding a wide variety of failure cases.

This research focuses on reducing failures caused by the existence of outdated map features in reused maps during stochastic KV-SLAM operations. There are three critical issues which appear in systems which do not handle outdated map points sufficiently well. The first issue relates to the size of the map, as only adding newly observed features without removing outdated ones results in a monotonically increasing map size. This puts an ever-increasing strain on compute resources to deal with the large map, and increases the complexity of optimization steps. The second issue is in regard to relocalization, the process by which a KV-SLAM system redetermines its position within the map after being lost. The process of relocalization starts with identifying candidate positions within the map which are visually similar to the current camera image. A high proportion of outdated map points interferes with this candidate selection, as the map's understanding of the visual features of the environment no longer match reality. The final issue affects the RANSAC process, a function which is run regularly as part of the process of determining the 3D transformation between successive frames. The RANSAC algorithm selects a random sample of points from the map and the new image to be localized, and attempts to determine a consistent 3D transformation which explains the correspondence. High proportions of outdated map points increase the risks of the RANSAC algorithm failing to find a consistent pose for the new frame, even if one exists. The map representation, relocalization, and RANSAC will be covered in greater detail in Section \ref{sec:kv_slam_background}.

There are two ways to prevent the failure modes specific to outdated map points in reused maps. The first option is to proactively prevent map features which have potential to become outdated (i.e. they are located on an object that can move) from ever entering the map. The second method is to retroactively identify and remove outdated features from the map once it is loaded. The mechanism at play in the proactive case tends to be semantic identification and segmentation of the input images, allowing  a process which requires significant compute resources, making it unsuitable for many compute constrained applications. The semantic and change-detection methods will be discussed in greater detail in Section \ref{sec:related_work}.

To this end, numerous methods for identifying and filtering non-static data from SLAM systems have been implemented to great success, allowing SLAM to perform well in highly dynamic environments. These implementations will be discussed in detail in Section \ref{sec:related_work}. A common theme amongst many of the best performing systems is the utilization of a learning model for identifying dynamic objects in an image frame, and ignoring any image features detected on them. This is effective, as it generates a map consisting entirely of what should be static objects, unlikely to change or move. The downside of this methodology is that it requires compute resources capable of running image segmentation and semantic identification in real time and in parallel with the other SLAM processes. Other methods of dynamic data removal utilize probability modeling, treating each image feature as an element which has some probability of existing. These methods have also seen success in improving operations in dynamic environments, but are also limited. Of the implementations which will be discussed in Section \ref{related_work}, not all maintain an understanding of the directions from which a point is observable, and none maintain a full history of a points directional observability. This can lead to points being culled from the map while they are temporarily obscured, or maintained long after they should have been culled.

% Describe why the robotics world would benefit from SLAM systems that can operate over long periods of time in changing environments.

Despite the challenges and risks associated with failure, research into improving SLAM's capabilities to long time horizons and more dynamic environments is extremely active. This is because of the significant benefits afforded by SLAM to operate in arbitrary environments, generate high quality maps, and produce accurate localization estimations. These capabilities are critical for integrating robotic autonomy into everyday life. In the specific domain of long-term, multi-session operations in intermittently static environments such as warehouses, factories, and even the ISS, improvements to SLAM's ability to localize and map move the technology closer to widespread adoption.  The work presented here intends to push the field slightly towards that goal by improving localization within reused maps by identifying and eliminating outdated map points.

% Describe what causes SLAM to struggle in long-term, multi-session operations in changing environments

The concept of multi-session operations involves a SLAM system which only operates intermittently. The primary use case for this class of system would be for care taking operations in environments which are only sometimes inhabited by people, E.g. factories during business hours.

% Describe the problems with existing map point culling methods, and why they are bad for low-compute platforms

