\subsection{Objectives and Scope}
\label{objectives}

The research presented in this thesis was inspired by and limited in scope to map reuse in stochastic keypoint-based visual SLAM operations. Limiting the scope does mean that some relevant topics will not be explored in depth, but will be mentioned in Section \ref{sec:future_work} as potential areas for follow-on research. The justification for the scope selection is given here.

The methods presented in this research will not work for all SLAM generated maps. While change detection techniques do exist for dense maps \cite{PLACEHOLDERa}, the method presented here would not be suitable, 

SLAM is extremely flexible in terms of what sensor inputs it can use, and the methodologies used to run it. But for the cases laid out so far such as the ISS, the desired system operates more as a self-updating localizer tha

This research has two primary objectives. The first is the implementation a novel map point removal method which considers the historical point-level viewpoint-dependent observability of map points into the decision to keep or discard points. The second is the integration of this map point removal method into an existing KV-SLAM system. Both objectives have several sub-goals, which are detailed below.

\subsubsection*{Objective 1: Implementation of a Viewpoint-Dependent Map Point Removal Method}

\subparagraph{Goal 1: Speed}
\subparagraph{Goal 2: Size}
\subparagraph{Goal 3: Accuracy}

\subsubsection*{Objective 2: Integration into an Existing KV-SLAM System}

\subparagraph{Goal 1: Maintain Performance in the General Case}
\subparagraph{Goal 2: Improve Performance When Using Outdated Maps}
\subparagraph{Goal 3: Maintain Computational Requirements}



It should be noted that the focus of this research is exclusively on KV-SLAM. Many other SLAM imlementations exists, capapble of using a wide variaety of sensors

% Define the scope of this work, focusing exclusively on map reuse between runs of SLAM in semi-static environments.

This research is intended to improve keypoint-based visual SLAM performance in situations where a pre-generated map is loaded to lock in a known coordinate frame, and in changing environments. The key distinction from other operation modes is the repeated operations in the same environment, with an expectation of the maintenance of large scale static features, but without the expectation for smaller visual features to remain the same between runs. This is accomplished through the introduction of a viewpoint-aware probabilistic model for the existence of map points. This research studies the effects of removing map points from pre-generated maps using this model, and the downstream effects on optimization, relocalization and map accuracy.

% Define the objectives of the model; to be lightweight and effective at identifying outdated map points, and to be usable in real-time SLAM systems.

The objective of the model is to produce an accurate method of representing the global observability of each map point in a map. The model should be lightweight, requiring only small amounts of additional space per map point. The model should integrate all observational data into its understanding of the map point's observability, while rejecting redundant data. Identification of map points which are outdated should be fast and accurate. Map points which remain visible should be robustly maintained.

% Define the objectives of the implementation, not to hurt performance in static environments, and to improve performance with previously generated maps in semi-static environments.

This model will be integrated into ORB-SLAM3 for testing purposes, ensuring we meet the implementation objectives of speed, accuracy and improvement. The model should not introduce significant increases in compute requirements. Additionally, performance in static environments should remain the same regardless of model implementation. However, under the specific scenario of map reuse in a changing environment, performance (as measured by trajectory accuracy to ground truth) should be improved.