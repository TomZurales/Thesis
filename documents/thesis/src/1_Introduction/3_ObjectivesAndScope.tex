\subsection{Objectives and Scope}
\label{objectives}

The research presented in this thesis was inspired by and limited in scope to map reuse in stochastic keypoint-based visual SLAM operations. Limiting the scope does mean that some relevant topics will not be explored in depth, but will be mentioned in Section \ref{sec:future_work} as potential areas for follow-on research. The justification for the scope selection is given here.

The methods presented in this research will not work for all SLAM systems, or on all SLAM generated maps. While change detection techniques do exist for dense maps \cite{PLACEHOLDERa}, the method presented here would not be suitable, as it relies on locating the same map features at different points in time. Dense maps identify matching positions through geometric correspondence, as opposed to feature descriptors. This applies to LiDAR-based SLAM systems as well, as no descriptor is generated for each point. The method which will be presented relies on the capability to apply and persist metadata to a particular map point, something which is not possible without point descriptors, leaving keypoint-based visual SLAM as the only option for this work.

The focus map reuse and the stochastic operating model can be motivated by the use cases discussed previously. On the ISS, the ideal navigation system would be a self-updating localizer than a SLAM system. The primary difference being that a localizer utilizes the same map and reference frame for all operations, while a SLAM system generates and use its own map. SLAM's features are extremely useful for exploration of unknown areas, but less so for repeated operations in known areas. The combination of map reuse and the stochastic operating model allow the system to act as a self-updating localizer, maintaining a map which is up-to-date with the last observed state of the world, while preserving as much previously collected data as possible.

Within this scope, we define two primary objectives. The first is the implementation a novel map point removal method which considers the historical point-level viewpoint-dependent observability of map points into the decision to keep or discard points. The second is the integration of this map point removal method into an existing KV-SLAM system. Both objectives have several sub-goals, which are detailed below.

\subsubsection*{Objective 1: Implementation of a Viewpoint-Dependent Map Point Removal Method}

\subparagraph{Goal 1: Speed}
\subparagraph{Goal 2: Size}
\subparagraph{Goal 3: Accuracy}

\subsubsection*{Objective 2: Integration into an Existing KV-SLAM System}

\subparagraph{Goal 1: Maintain Performance in the General Case}
\subparagraph{Goal 2: Improve Performance When Using Outdated Maps}
\subparagraph{Goal 3: Maintain Computational Requirements}
