\subsection{Research Questions}

\subsubsection*{Question 1: How can the historical viewpoint-dependent observability of points be modeled in order to meet the objectives laid out in Objective 1?}

To integrate historical viewpoint-dependent observability into existing change detection approaches requires a compact model of observability at the point level. Three model implementations are explored in Section \ref{sec:observability_models}, and evaluated in Section \ref{sec:observability_model_eval}.

\subsubsection*{Question 2: How does the integration of historical observability data affect the estimation of point existence confidence?}

Each of the three model implementations is integrated into an existing change detection model, discussed in Section \ref{sec:existence_confidence}, which utilizes a Bayesian update step to iteratively update a confidence estimation of the map point's existence. Each of the three models is evaluated for speed, accuracy and robustness in Section \ref{sec:existence_confidence_eval}.

\subsubsection*{Question 3: Can the change detection model be integrated into an existing KV-SLAM system in a way that meets the goals of Objective 2?}

The extended confidence update is integrated into ORB-SLAM3, a popular KV-SLAM system, for evaluation against a series of datasets intended to test the system across a range of simple and challenging situations. The integration is discussed in Section \ref{sec:orb_slam3_integration}