\subsection{Problem Context}

% Talk about time on the astrobee project and the problems encountered

The International Space Station (ISS) has three robotic passengers on board. These robots, known as Astrobees, were launched with the intention of being a platform for zero-gravity robotics research, a camera rig for recording astronaut activity, and a sensor platform for performing station keeping and monitoring tasks \cite{smithASTROBEENEWPLATFORM}. Core to each of these missions is the ability for Astrobee to autonomously localize and navigate throughout the U.S. Orbital Segment (USOS). Unfortunately, the harsh realities of navigation on board the ISS quickly became apparent. The high-level structure of the station is constant, but the visual features within the station are not stable, changing due to equipment resupplies, hardware movements, astronaut activity, and lighting changes. The irregularity in the visual features of the ISS degraded the performance of the Astrobee localizer, which relied on visual features stored in a pre-generated map \cite{soussanAstroLocEfficientRobust2022}. Mapping the ISS required input from ground crew and astronauts, which, due to the tight timetables associated with ISS operations, led to infrequent map updates. These outdated maps caused frequent lost states, inaccurate localization estimations, and overall poor performance of the navigation stack. Eventually, Astrobee activities were restricted almost entirely to the Japanese Experiment Module (JEM), a relatively small area which could be kept visually stable, and could be quickly remapped when needed \cite{carlinoLessonsLearnedAstrobee}.

A project was proposed to augment Astrobee's navigation stack with a Keypoint-Based Visual SLAM (KV-SLAM) system. The idea behind this change was to have the SLAM system load an initial map from the previous mission, localize within this map, then update the map with the latest visual features from the environment. At the end of the operation, the updated map could be saved off for use in the next mission \cite{zuralesCollaborativeSensingMapping2024}. For this research, we will call this operational concept of iteratively updating previous maps between operations \textit{episodic SLAM}. Testing of the SLAM system showed presented the same issues which were encountered by the Astrobee localizer when attempting to reuse outdated maps. The conjecture stemming from these results was that for reused maps to be useful, they must not only be updated with the latest visual features, but they must also remove features which have become outdated.

Three key lessons were learned from the Astrobee work which motivated the research presented here. First, the episodic operating model has potential in a wide variety of applications, providing a method for regular, large scale environmental monitoring of changing environments such as factories, warehouses and offices. Second, map reuse is a powerful tool for robotic systems operating in the same environment over long periods of time, allowing for the preservation of structural map data and the global reference frame. Finally, SLAM systems are not inherently robust to outdated maps, showing degraded performance as the initial map diverges from the true environment.
