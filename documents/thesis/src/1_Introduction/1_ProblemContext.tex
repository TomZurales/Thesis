\subsection{Problem Context}

% Talk about time on the astrobee project and the problems encountered

The International Space Station (ISS) has three robotic passengers on board. These robots, known as Astrobees, were launched with the intention of being a platform for zero-gravity robotics research, a camera rig for recording astronaut activity, and a sensor platform for performing station keeping and monitoring tasks \cite{smithASTROBEENEWPLATFORM}. Core to each of these missions is the ability for Astrobee to autonomously localize and navigate throughout the U.S. Orbital Segment (USOS). Unfortunately, the harsh realities of navigation on board the ISS quickly became apparent. The high-level structure of the station is constant, but the visual features within the station are not stable, changing due to equipment resupplies, hardware movements, astronaut activity, and lighting changes. The irregularity in the visual features of the ISS degraded the performance of the Astrobee localizer, which relied on visual features stored in a pre-generated map \cite{soussanAstroLocEfficientRobust2022}. Mapping the ISS required input from ground crew and astronauts, which, due to the tight timetables associated with ISS operations, led to infrequent map updates. These outdated maps caused frequent lost states, inaccurate localization estimations, and overall poor performance of the navigation stack. Eventually, Astrobee activities were restricted almost entirely to the Japanese Experiment Module (JEM), a relatively small area which could be kept visually stable, and could be quickly remapped when needed \cite{carlinoLessonsLearnedAstrobee}.

A project was proposed to augment Astrobee's navigation stack with a keypoint-based visual SLAM system. The idea behind this change was to have the SLAM system load an initial map from a previous sortie, localize within this map, then update the map with the latest visual features from the environment. At the end of the sortie, the updated map could be saved off for use in the next mission \cite{zuralesCollaborativeSensingMapping2024}. For this research, we will call this operational concept of iteratively updating previous maps between operations \textit{stochastic SLAM}. The SLAM system

% Describe what bad things happen when SLAM systems fail

During ISS operations, the Astrobee robots would regularly lose localization. This was primarily caused by the localizer's sensitivity to lighting changes and equipment movement, leading to the Astrobee being restricted to operations within the Japanese Experiment Module (JEM) \cite{carlinoLessonsLearnedAstrobee}. Trust in autonomous systems is precarious, quickly falling upon the observation of failure \cite{robinetteEffectRobotPerformance2017}. Failures in SLAM can result in mislocalization, leading to task failure, navigational instability, or the possibility of physical damage to the robotic platform\cite{nahavandiComprehensiveReviewAutonomous2025a}. Therefore, in order for autonomous SLAM based navigation to see widespread use, it must be accurate in its estimations, and robust to a wide variety of failure cases.

% Describe why the robotics world would benefit from SLAM systems that can operate over long periods of time in changing environments.

Despite the challenges and risks associated with failure, research into improving SLAM's capabilities to long time horizons and more dynamic environments is extremely active. This is because of the significant benefits afforded by SLAM to operate in arbitrary environments, generate high quality maps, and produce accurate localization estimations. These capabilities are critical for integrating robotic autonomy into everyday life. In the specific domain of long-term, multi-session operations in intermittently static environments such as warehouses, factories, and even the ISS, improvements to SLAM's ability to localize and map move the technology closer to widespread adoption.  The work presented here intends to push the field slightly towards that goal by improving localization within reused maps by identifying and eliminating outdated map points.