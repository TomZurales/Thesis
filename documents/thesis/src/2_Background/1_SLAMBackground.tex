\subsection{SLAM}

% What is SLAM? What are its goals? What are its use cases?
SLAM refers to the joint problem of simultaneously generating a map of an environment and estimating the position of an observer within that map based on sensor observations. The benifit of SLAM is the ability to operate with no a-priori information, instead incrementally creating and updating the map of the environment.

% Quick discussion of SLAM modalities to motivate discussion on keypoint SLAM.
There are many sensors which are useful for SLAM, each having their own strengths and weaknesses depending on the situation. The decision of which sensor modality to use in a given situation depends on the performance requirements of the system, and the environmental conditions where the system will be operating.

\subsubsection{Keypoint-Based Visual SLAM}

% Definition and high-level overview
Keypoint-based visual SLAM refers to the subset of SLAM systems which utilize keypoints extracted from images as the primary source of information for mapping and localization. Keypoints are distinctive visual features that can be reliably and repeatedly detected and differentiated in different image frames. Keypoints are used to establish correspondences between successive image frames, allowing the motion of the camera to be estimated by determining transformations that explain the observed changes in keypoint positions.

% 

% Use Cases and Applications

% Advantages and Limitations