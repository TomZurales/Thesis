\subsection{SLAM}

SLAM refers to the joint problem of simultaneously generating a map of an environment and estimating the position of an observer within that map based on sensor observations. There are numerous sensor modalities that can be used for SLAM, including

\subsubsection{Keypoint-Based Visual SLAM}

% Definition and high-level overview
Keypoint-based visual SLAM refers to the subset of SLAM systems which utilize keypoints extracted from images as the primary source of information for mapping and localization. Keypoints are distinctive visual features that can be reliably and repeatedly detected and differentiated in different image frames. Keypoints are used to establish correspondences between successive image frames, allowing the motion of the camera to be estimated by determining transformations that explain the observed changes in keypoint positions.

% 

% Use Cases and Applications

% Advantages and Limitations