\subsection{Keypoint-Based Visual SLAM}

% What is SLAM? What are its goals? What are its use cases?
SLAM refers to the joint problem of simultaneously generating a map of an environment and estimating the position of an observer within that map based on sensor observations. With research beginning in the 1980s \cite{smithEstimatingUncertainSpatial1988}, SLAM has become a de facto standard in robotics for tasks which require operations within unfamiliar environments. Through careful selection of sensors, SLAM systems can operate in a wide variety of environments, from indoor office spaces to densely forested areas, and even underwater or in space. The benefit of SLAM when compared to other navigation methods such as pure localization is the ability to operate with no a-prioiri information, incrementally creating and updating the map, and the estimation of the observer's position within it. These features make SLAM a powerful too for autonomous navigation, leading to its widespread use in robotics, augmented reality, and autonomous vehicles.

% High level overview of the challenges of SLAM
However, there are challenges associated with SLAM. Despite numerous advancements, the computational complexity of performing SLAM in real-time remains a challenge, meaning that on many resource constrained platforms, implementations can struggle to keep up with the rate of sensor data acquisition \cite{semenovaQuantitativeAnalysisSystem2022}. Sensor selection greatly impacts the environments in which SLAM will perform well. For example, LIDAR-based SLAM systems excel in low-light conditions and environments with low visual texture, but struggle in environments with reflective surfaces or low geometric complexity \cite{khanComparativeSurveyLiDARSLAM2021}, while the opposite is true for visual SLAM systems \cite{camposORBSLAM3AccurateOpenSource2021a}. A table comparing the advantages and disadvantages of various SLAM sensor modalities is provided in Table \ref{tab:slam_sensor_modalities}. Additional challenges arise as the timeframe and scale of the SLAM operation increases. The field studying methods of extending SLAM's performance into larger environments and timespans is known as life-long SLAMBecause there is no fixed reference to compare against, SLAM systems can suffer from sensor error accumulation, leading to inaccurate maps and poses over time \cite{cadenaPresentFutureSimultaneous2016}. 

\begin{table}[ht!]
    \centering
    \caption{Comparison of SLAM sensor modalities}
    \label{tab:slam_sensor_modalities}
\end{table}