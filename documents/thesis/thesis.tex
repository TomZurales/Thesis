\documentclass[12pt]{article}

\usepackage{graphicx}
\usepackage[margin=1in]{geometry}
\usepackage{setspace}

\title{Thesis}
\author{Tom Zurales}
\date{July 2025}

\begin{document}
\doublespace

\maketitle

\newpage

\begin{abstract}
  This research details the implementation and analysis of a novel, viewpoint aware method for map point culling for use in keypoint based visual SLAM systems. This method makes use of perspective dependent shells around each map point, allowing for the storage of overall observability metadata using constant additional space. This metadata allows for the overall probability of a point's existence to be continuously calculated using a simple Bayesian update step. This existence probability can be used in a myriad of ways. This research explores its use as a method of culling outdated map points, such as those originally seen on an object which has since moved, and as an extension to the RANSAC algorithm, providing the ability to select more robust map points. We provide the implementation of the perspective aware metadata shell as an open-source library, as well as an ORB\_SLAM3 implementation utilizing the library for point culling and RANSAC improvement. Additionally, a set of co-registered visual-inertial-lidar datasets are released, containing scenarios specifically intended to exercise and characterize the performance of the system. Through our analysis, (discuss effects on well known, non-dynamic SLAM datasets, along with my datasets)
\end{abstract}

\newpage

\tableofcontents

\section{Introduction}

The research presented in this thesis details the conception, implementation and analysis

\subsection{Motivation}

% Describe the project that inspired this work; astrobee map reuse on the ISS.

The inspiration for this research came from a project with the goal of augmenting the Astrobee robot navigation system on the International Space Station (ISS) with keypoint-based visual SLAM. At the time of the project, Astrobee navigation consisted of localization on a pre-generated map, which required both astronaut and ground team time to produce, leading to infrequent map updates \cite{soussanAstroLocEfficientRobust2022}. By augmenting the Astrobee with SLAM, we aimed to reduce the need for manual map updates and improve the robot's autonomous capabilities. Additionally, we planned to reuse maps between Astrobee operation sessions to produce consistent localization estimates, registered to the established coordinate frame of the ISS. However, we found that the ISS presents unique challenges for robot navigation. The structure of the station is stable, but the contents and position of equipment changes frequently. Consequently, maps created aboard the ISS quickly become obsolete. Attempting to load outdated maps for reuse degraded Astrobee's navigation performance, limiting the robot's ability to perform autonomously and complete its mission objectives \cite{zuralesCollaborativeSensingMapping2024}.

% Describe why the robotics world would benefit from SLAM systems that can operate over long periods of time in changing environments.

This failure of autonomy for long-term, multi-session operations in changing environments presented a problem for Astrobee, just as it does for many other robotics applications. If solved to the point where SLAM systems could operate robustly, over long periods of time, and in highly dynamic environments, integration of robotics into human environments would be greatly simplified.

\subsection{Objectives and Scope}

% Define the objectives of this research. What are the overall goals. What should be different once this is implemented

% Describe the scope of the research; what limits are placed on the objectives by this scope?


This research 

This research intends to build upon the previously developed probability models, in order to distill the update step of each map point's probability of existence into a simple Bayesian update step. The goals for this model are as follows:

1. To utilize constant additional space for each map point
2. To complete the update step in constant time
3. To resist updating confidence levels with redundant data

\subsection{Contribution}

Through this research, we introduce an incrementally updated directional confidence model for the existence of map points. This model differs from other point removal optimizations in several ways. First, this implementation avoids the use of neural networks, facilitating use on resource constrained hardware without facilities optimized to run them. Second, while other probability based point removal optimizations have been developed, this model introduces the idea of utilizing a continuously updated perspective dependent shell of metadata for each keypoint, which can be used to reduce the problem of point existence to a simple Bayesian update step. This implementation allows perspective of observation to play a role in the point's existence probability update step, and avoids some of the common a priori work such as prior estimation common to other point removal optimization techniques. This shell is implemented in both finite and continuous modalities, utilizing regular convex polyhedral shells in the finite case, and von-meiser fisher distributions on the sphere in the continuous case.

To facilitate future research, this model is released as an open-source library, which is compatible with any keypoint based visual SLAM implementation. Additionally, a collection of co-registered visual-inertial and LIDAR datasets is provided, containing instances of multiple traversals through the same environments with changes to scene contents. Information regarding the locations of these environmental changes is included in the dataset, facilitating the benchmarking of point removal optimization implementations.

\input{src/1_Introduction/4_RoadMap}

\chapter{Background}

In this chapter, we provide a high level overview of the definitions, objectives, and history of SLAM, in addition to an overview of the common sensor modalities found in SLAM systems. Following this, we discuss the stages of the SLAM pipeline in the generic case, followed by a more in-depth exploration of keypoint-based visual SLAM, the sensor modality targeted by this research. Next, we look into several of the widely adopted extensions to the base SLAM pipeline which address core issues and enhance performance. A discussion of extensions which have similar goals to this research is held for the chapter on related works.

\subsection{SLAM Overview}

This research is acts as an extension to keypoint-based visual SLAM; a term which warrants some explanation. But before exploring the specifics of keypoint-based visual SLAM, some background on the general SLAM problem is required. The idea behind SLAM is to simultaneously produce a map of an environment, and determine the position of the observer within the map based on a set of sensor data. The process differs heavily based on the sensor types being utilized. For example, LIDAR provides a direct measurement of 3D distances from the sensor, while an RGB camera must calculate them from correspondences between multiple frames. While implementations differ heavily, a common SLAM pipeline could be described as follows:

\subsubsection{The SLAM Pipeline}

This stage is responsible for the creation of the initial map.

There have been hundreds of SLAM implementations for a wide variety of sensors, commonly targeting combinations of monocular, stereo or RGBD cameras, IMUs, LIDARs, etc.

Due to it providing the motivation for this project, the Astrobee robots will me mentioned several times throughout this work. The Astrobee project was motivated by the desire to research human/robot interaction, robotic automation and inspection, and to provide a research platform on which companies and researchers could deploy software and hardware for testing in a micro-gravity environment. The Astrobee platform has been used to develop satellite rendezvous control algorithms, grippers to capture tumbling orbital debris, inspection methods to autonomously detect anomalous operation, and many other space habitation focused endeavors.

\subsubsection{Keypoint-Based Visual SLAM}

The term Keypoint-Based Visual SLAM refers to the SLAM modality which primarily utilizes key points extracted from images as the primary means of mapping and navigating. This is distinct from systems like LIDAR-based SLAM, which utilize direct distance measurements from a LIDAR sensor, or Dense Visual SLAM, which

\subsubsection{Extensions to Core SLAM}

\subsection{Additional Fields of Research}

\subsubsection{Directional Probability}

\chapter{Related Work}
\label{related_work}

Numerous methods have been developed to improve long-term SLAM performance, generalizing the problem to remove the constraint that the environment remains static.
https://arxiv.org/pdf/2209.10710 - ChangingSlam - Uses a Bayes filter to remove changing map points. Utilizes semantic detection to determine dynamic objects before filter is used, which prevents it from working on deformable objects.

\subsection{Point Removal Optimizations}

\subsubsection{Semantics Based Implementations}

\subsubsection{Probability Based Optimizations}

\include{src/Implementation}

\section{Experimental Analysis}

\subsection{Dataset Creation}

\subsubsection{Hardware}

\subsubsection{Structure}

\subsubsection{Dataset Overview}

\subsection{Evaluation Metrics}

\subsection{SLAM System Configurations}

\subsubsection{Parameter Tuning}

\subsection{Results}

\subsubsection{Quantitative Evaluation}

\subsubsection{Qualitative Evaluation}

\subsubsection{Ablation Study}


\include{src/Discussion}

\include{src/Conclusion}

\include{src/Appendices}

\end{document}