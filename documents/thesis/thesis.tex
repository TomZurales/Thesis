\documentclass[12pt]{article}

\usepackage{graphicx}
\usepackage[margin=1in]{geometry}
\usepackage{setspace}
\usepackage[utf8]{inputenc}
\usepackage[backend=biber]{biblatex}
\usepackage{amsmath}
\usepackage{amssymb}
\usepackage{amsthm}
\usepackage{titlesec}
\usepackage{dirtytalk}
\usepackage{enumitem}
\usepackage[font=small,labelfont=bf]{caption}

\DeclareMathOperator*{\argmin}{argmin}
\captionsetup[figure]{hangindent=0pt, singlelinecheck=false, width=0.9\textwidth}
\setlist[itemize]{noitemsep} 

\titleformat{\paragraph}[block]{\normalfont\normalsize\bfseries}{\theparagraph}{1em}{}[]

\newcommand{\definition}[2]{\textbf{#1}: #2}

\titleformat{\subparagraph}[block]{\normalfont\normalsize\itshape}{\thesubparagraph}{1em}{}[]

\addbibresource[location=remote]{http://127.0.0.1:23119/better-bibtex/export?/library;id:1/collection;key:GT493ARA/Thesis.biblatex}

\title{Thesis}
\author{Tom Zurales}
\date{July 2025}

\begin{document}
\doublespace

\maketitle

\newpage

\begin{abstract}
    This research details the implementation and analysis of a novel, viewpoint aware method for map point culling for use in keypoint based visual SLAM systems. This method makes use of perspective dependent shells around each map point, allowing for the storage of overall observability metadata using constant additional space. This metadata allows for the overall probability of a point's existence to be continuously calculated using a simple Bayesian update step. This existence probability can be used in a myriad of ways. This research explores its use as a method of culling outdated map points, such as those originally seen on an object which has since moved, and as an extension to the RANSAC algorithm, providing the ability to select more robust map points. We provide the implementation of the perspective aware metadata shell as an open-source library, as well as an ORB\_SLAM3 implementation utilizing the library for point culling and RANSAC improvement. Additionally, a set of co-registered visual-inertial-LiDAR datasets are released, containing scenarios specifically intended to exercise and characterize the performance of the system. Through our analysis, (discuss effects on well known, non-dynamic SLAM datasets, along with my datasets)
\end{abstract}

\newpage

\tableofcontents

\newpage

\listoffigures

\addcontentsline{toc}{section}{Definitions}
\subsection*{Definitions}

{\large \textbf{Keypoint-Based Visual SLAM}} - SLAM

\section{Introduction}

The research presented in this thesis details the conception, implementation and analysis

\subsection{Motivation}

% Describe the project that inspired this work; astrobee map reuse on the ISS.

The inspiration for this research came from a project with the goal of augmenting the Astrobee robot navigation system on the International Space Station (ISS) with keypoint-based visual SLAM. At the time of the project, Astrobee navigation consisted of localization on a pre-generated map, which required both astronaut and ground team time to produce, leading to infrequent map updates \cite{soussanAstroLocEfficientRobust2022}. By augmenting the Astrobee with SLAM, we aimed to reduce the need for manual map updates and improve the robot's autonomous capabilities. Additionally, we planned to reuse maps between Astrobee operation sessions to produce consistent localization estimates, registered to the established coordinate frame of the ISS. However, we found that the ISS presents unique challenges for robot navigation. The structure of the station is stable, but the contents and position of equipment changes frequently. Consequently, maps created aboard the ISS quickly become obsolete. Attempting to load outdated maps for reuse degraded Astrobee's navigation performance, limiting the robot's ability to perform autonomously and complete its mission objectives \cite{zuralesCollaborativeSensingMapping2024}.

% Describe why the robotics world would benefit from SLAM systems that can operate over long periods of time in changing environments.

This failure of autonomy for long-term, multi-session operations in changing environments presented a problem for Astrobee, just as it does for many other robotics applications. If solved to the point where SLAM systems could operate robustly, over long periods of time, and in highly dynamic environments, integration of robotics into human environments would be greatly simplified.

\subsection{Objectives and Scope}

% Define the objectives of this research. What are the overall goals. What should be different once this is implemented

% Describe the scope of the research; what limits are placed on the objectives by this scope?


This research 

This research intends to build upon the previously developed probability models, in order to distill the update step of each map point's probability of existence into a simple Bayesian update step. The goals for this model are as follows:

1. To utilize constant additional space for each map point
2. To complete the update step in constant time
3. To resist updating confidence levels with redundant data

\subsection{Contribution}

Through this research, we introduce an incrementally updated directional confidence model for the existence of map points. This model differs from other point removal optimizations in several ways. First, this implementation avoids the use of neural networks, facilitating use on resource constrained hardware without facilities optimized to run them. Second, while other probability based point removal optimizations have been developed, this model introduces the idea of utilizing a continuously updated perspective dependent shell of metadata for each keypoint, which can be used to reduce the problem of point existence to a simple Bayesian update step. This implementation allows perspective of observation to play a role in the point's existence probability update step, and avoids some of the common a priori work such as prior estimation common to other point removal optimization techniques. This shell is implemented in both finite and continuous modalities, utilizing regular convex polyhedral shells in the finite case, and von-meiser fisher distributions on the sphere in the continuous case.

To facilitate future research, this model is released as an open-source library, which is compatible with any keypoint based visual SLAM implementation. Additionally, a collection of co-registered visual-inertial and LIDAR datasets is provided, containing instances of multiple traversals through the same environments with changes to scene contents. Information regarding the locations of these environmental changes is included in the dataset, facilitating the benchmarking of point removal optimization implementations.

\input{src/1_Introduction/4_RoadMap}

\section{Background}
\label{background}

In this chapter, we provide a high level overview of SLAM, with special focus paid to the Keypoint-Based Visual SLAM modality, and the ORB-SLAM3 implementation. This research uses ideas from the field of Directional Statistics, and so we provide a brief overview of the subject, and the Von-Mises Fisher distribution as well.

\subsection{SLAM Overview}

% What is SLAM? What are its goals? What are its use cases?
SLAM refers to the joint problem of simultaneously generating a map of an environment and estimating the position of an observer within that map based on sensor observations. The benefit of SLAM is the ability to operate with no a-priori information, incrementally creating and updating the map of the environment. This is in contrast to pure localization which requires foreknowledge of the environment where it is operating. The ability to map and localize in arbitrary environment has applications in robotics, alternate and virtual reality, self-driving vehicles, and many other fields. Environmental observations can be collected through a variety of sensors, the selection of which depends heavily on the system's requirements, and the intended environment of operation.

% Quick discussion of SLAM modalities to motivate discussion on keypoint SLAM.
The complement of sensors in a SLAM system defines its 'sensor modality'. This selection has great effect on the system's accuracy and robustness, and plays a part in determining the types of environment in which the system will perform well or struggle.

\subsubsection{A Comparison of SLAM Sensor Modalities}

% Quick description of other SLAM sensor modalities; LIDAR, RGBD, etc
While more esoteric implementations exist, the primary sensor of a SLAM system is typically a camera sensor, Light Detection and Ranging (LiDAR) sensor, or a combination camera + depth (RGBD) sensor. LiDAR sensors take direct 3D distance measurements of the environment by measuring the return time of emitted laser pulses. These sensors produce highly accurate 3D maps in the form of point clouds, but do not collect visual information from the environment. In contrast, RGBD sensors utilize a combination of an image sensor and an array of distance sensors to generate an RGB image co-registered with a depth map. Again, these sensors take direct 3D measurements of the environment (often at a lower resolution that pure LiDAR), but also have access to the visual data of the environment.

% Quick description of other visual modalities; stereo, mutli-camera
Unlike LiDAR and RGBD sensors, cameras are incapable of directly measuring 3D distances. Instead, 3D data must be extrapolated from multiple images taken from different viewpoints. SLAM systems utilize 2D-2D image correspondences to determine an initial 3D map, then 2D-3D matches to determine the pose of new images within the map. The camera-based sensor modality can be broken down by the number of cameras used, and their position on the platform. The stereo modality involves two camera sensors placed in such a way that their fields of view (FOVs) overlap. Within this shared field of view, the fixed distance (baseline) between the cameras allows depths to be estimated by measuring the disparity between corresponding points in the images. In the monocular case, there is no known baseline, therefore motion is required to obtain multiple views in order to triangulate an initial 3D estimate of the environment. The multi-camera modality has two or more cameras with FOVs that may or may not overlap.

Visual methods can again be broken down into direct and keypoint-based methodologies. The distinction here is in regard to how the image data is used by the slam system as opposed to what sensor is selected. The direct method uses raw pixel values to determine the motion of the camera, similar to optical flow. This method is typically more computationally intense than keypoint-based methods, but can produce more accurate results in environments with low visual texture. On the other hand, keypoint-based methods extract sparse, distinctive features from images, and find correspondences between the same features in different frames to estimate motion. The maps generated by direct methods tend to be dense, meaning a distance is estimated for every pixel in the image, while keypoint-based methods generate sparse maps, consisting only of keypoints extracted from the images.

% Discussion of the advantages and disadvantages of other sensor modalities
Direct 3D measurements simplify the SLAM problem greatly when compared to pure visual methods. However, there are advantages and disadvantages to each sensor modality, and therefore the selection of which to use in a given situation will depend heavily on the requirements of the platform running SLAM, and the environment in which it will operate. LIDAR offers accurate 3D geometry measurements, robust to lighting conditions, but struggle in environments with low geometric complexity (Eg. a long rectangular hallway) or reflective objects. This modality is not practical in many applications due to its cost, size and power requirements. Additionally, LiDAR sensors (excluding solid state LIDAR) often rotate rapidly which produces a torque moment that can be difficult to manage in small robot platforms. These sensors offer significant performance to platforms which can use them, but do not take advantage of the rich visual information of the environment. The RGBD sensor combines an RGB camera with a co-registered array of depth sensors or a solid state LiDAR. These sensors have the ability to take direct 3D measurements of the environment within their field of view, and can utilize visual features to provide semantic identification, object tracking, image segmentation, etc. There are many options for RGBD cameras, offering the flexibility to select a sensor which will meet power, size and weight requirements, leaving cost as the primary disadvantage. The camera modality is by far the cheapest and most flexible, making it ideal for small robotic platforms and VR/AR applications. The trade-off comes in the form of additional software complexity and compute requirements to produce 3D estimates from 2D data. The camera sensor modality works best in environments with high visual texture, and will struggle to produce accurate measurements in low light conditions.

\subsubsection{The SLAM Data Pipeline}
The general path that sensor data takes through a SLAM system can be thought of as a pipeline, seen in Figure \ref{fig:general_slam_pipeline}

The following is an example for figures -

\begin{figure}[h!]
    \includegraphics[scale=1.7]{birds.jpg}
    \caption{General data flow through a SLAM system}
    \label{fig:general_slam_pipeline}
\end{figure}


\subsection{Enhancements}

\subsubsection{Keypoint-Based Visual SLAM}

% paragraph - Defining characteristics of Keypoint-Based Visual SLAM; using image features, and determining motion through keypoint correspondences

Keypoint-based visual SLAM refers to implementations which utilize keypoints extracted from images as the primary source of information for mapping and localization. Keypoints are sparse, distinctive visual features that can be reliably and repeatedly detected and differentiated between image frames. Keypoints are used to establish correspondences between successive image frames, allowing the motion of the camera to be estimated by determining transformations that explain the observed changes in keypoint positions.

% Revisitation of the SLAM data pipeline from the perspective of keypoint-based visual SLAM; talk about initialization through 2D-2D correspondences, pose estimation through 2D-3D correspondences

% Revisitation of how enhancements are implemented in keypoint based visual slam

\subsubsection{The SLAM Pipeline}

This stage is responsible for the creation of the initial map.

There have been hundreds of SLAM implementations for a wide variety of sensors, commonly targeting combinations of monocular, stereo or RGBD cameras, IMUs, LIDARs, etc.

Due to it providing the motivation for this project, the Astrobee robots will me mentioned several times throughout this work. The Astrobee project was motivated by the desire to research human/robot interaction, robotic automation and inspection, and to provide a research platform on which companies and researchers could deploy software and hardware for testing in a micro-gravity environment. The Astrobee platform has been used to develop satellite rendezvous control algorithms, grippers to capture tumbling orbital debris, inspection methods to autonomously detect anomalous operation, and many other space habitation focused endeavors.

\subsubsection{Keypoint-Based Visual SLAM}

The term Keypoint-Based Visual SLAM refers to the SLAM modality which primarily utilizes key points extracted from images as the primary means of mapping and navigating. This is distinct from systems like LIDAR-based SLAM, which utilize direct distance measurements from a LIDAR sensor, or Dense Visual SLAM, which

\subsubsection{Extensions to Core SLAM}

\subsection{Additional Fields of Research}

\subsubsection{Directional Probability}

\section{Related Work}
\label{sec:related_work}

This chapter contextualizes this thesis within the broader body of SLAM research. The topics covered here are chosen based on their relevance to the previously outlined objectives, specifically the identification and removal of data that could cause instability or errors. While the goals of this research align closely with semantic identification and lifelong SLAM, the methods of implementation and integration are  closely related to change detection, point stability and reobservability confidence. The following sections explore these topics in detail, comparing with the proposed goals and methods.
% \input{src/3_RelatedWork/1_LongTermSLAM.tex}

\section{Implementation}

\subsection{Method Overview}

This section details our method for achieving the goals laid out in Section \ref{objectives}. The primary objective is to identify map points which, while previously viewed, are no longer visible due to environmental changes. To address this, we assign an incrementally updated probability of existence value to each map point. Observations of a map point increase our overall confidence in its existence. Conversely, failure to observe a map point from a viewpoint where it \textit{should} have been visible lowers our confidence in its existence. Determining whether a map point should be observable from a given viewpoint is not trivial, motivating the need for a viewpoint-aware observability model.

\subsubsection{Viewpoint-Aware Observability of Map Points}

% define what we mean by viewpoint aware observability; a mapping between observation unit direction vector and seen/not seen binary observations

We define viewpoint aware observability as a function
\begin{align*}
    \boldsymbol{f:}\mathbb{S}^2\times\mathbb{R}^+\rightarrow\{0,1\}
\end{align*}
where $\mathbf{v}\in\mathbb{S}^2$

% Justify why distance needs to be accounted for, use a diagram to show that by including distance, we can account for occlusions in the model

This model of observability does not account for occlusions. Therefore, we instead utilize the function
$$
    \mathbb{S}
$$

% Say what we are trying to build; a model which can produce a probability that a point will be seen given an observation direction and a distance


\subsubsection{Observability Shell Representations}

\subsubsection{Existence Probability and Update Rule}

\subsubsection{Application to Point Pruning and Selection}


\section{Experimental Analysis}

\subsection{Evaluation Metrics}

\subsection{SLAM System Configurations}

\subsubsection{Parameter Tuning}

\subsection{Results}

\subsubsection{Quantitative Evaluation}

\subsubsection{Qualitative Evaluation}

\subsubsection{Ablation Study}


\section{Conclusion and Future Work}


\subsection{Future }

\addcontentsline{toc}{section}{References}
\printbibliography

\section*{Appendices}
\addcontentsline{toc}{section}{Appendices}

\subsection*{Algorithm Implementations}
\addcontentsline{toc}{subsection}{Algorithm Implementations}

\subsection*{Full Results}
\addcontentsline{toc}{subsection}{Full Results}
test

\end{document}