\section{Implementation}

A probability model for the elimination of inaccurate map points
For each new frame which comes in, the tracking thread completes its estimation of the new pose, based on the motion model, the local map, or the reference keyframe. This pose is then optimized to improve local consistency. Post localization, the estimation of the current pose will be as good as possible based on the currently available information. This estimated pose, along with the set of map points in the map are then processed to update each map point’s probability of existence. Step one involves projecting each point to the camera’s image plane, determining the set of map points which should be visible given the currently estimated pose. We can then split these “expected” map points into two groups: those that are seen, and those that are not seen. For the map points which are seen, we set their probability of existence to 1, since we are sure they still exist. Next, their icosFaceProbs is increased using Bayes theorem to indicate that the point is visible from that perspective. Finally, the nearest vertex in icosVertexDists is set to the current distance between the camera and the map point, if it’s larger than the current setting. If all map points associated with the face are 0, we do something else I guess. If the point is not seen, things are a bit more interesting. First of all, the icosFaceProbs is updated using Bayes theorem, this time decreasing it. If the point’s overall probability drops enough for a specific face, the vertex distances begin to drop, in the hopes of moving to the inside of any obstructions. This prevents observations from one perspective from having too much power over the overall probability of existence. The distance is dropped by ½, and the face probability is set to the average of the adjacent face’s probabilities.

\subsubsection{REWORDING}

For each observed map point, we require the ability to determine both the perspectives from which it can be observed, and the maximum distance from that perspective at which the point can be observed. This effectively eliminates instances where a point is observable from a perspective while in front of a wall, but not while behind the wall.
Icosahedron Observability Shell
An icosahedron was selected as the geometry for the observability model because it has the highest number of faces for a convex, regular polyhedra. Effectively, by using 12 parameters for vertex distances MAYBE CHANGE TO 20 FACE DISTANCES LATER and 20 parameters for face weights, we are able to estimate the probability of observing a map point from every possible perspective and distance. This probability is utilized in a Bayesian update step to modify the map points overall probability of existence.
Observability Shell Construction
A standard construction of an icosahedron utilizes three perpendicularly intersecting rectangles of ratio 1:(1+sqrt(5))/2. The vertices of these rectangles become the vertices of the icosahedron, and the edges are formed from the short edges of the rectangles, and regular triangles drawn between them. A consistent naming scheme is needed for the vertices and faces, and can be arbitrarily decided. For simplicity, rectangles lying on the coordinate planes and centered on the origin are utilized for the construction. ORB-SLAM3 utilizes a right handed coordinate system based on the first frame of a new map with x pointing right in the image plane, y pointing down, and z pointing forward. Figure XX shows the relationships between the coordinate frames, construction rectangles, vertex names, and face names utilized for this construction. Placing this icosahedron shell around a map point is accomplished by adding the map point’s global coordinates to the vertex coordinates.
Continuous Observability Shell
Evaluation
Simulation methods for the quantification of the effect of inaccurate maps on tracking and relocalization
Before implementing the probability model, significant work can be done through simulation to determine the degree to which an outdated map affects SLAM performance. I generated several 3D environments, and several visual-inertial SLAM datasets though these environments for the
Dataset generation
Dataset generation takes place in three steps. The first step is Path Recording, followed by Dynamics Recording, and finally, sensor generation. These steps are described in detail below.
Path Recording
In this step, a world (SDF) is selected, and the simulated robot is driven through it. At each frame update, the controls provided by the user are recorded for future playback. The output of this step is a directory structure containing a copy of the world, and the file containing the recorded path.
Dynamics Recording
At this stage, the world is reloaded, and the paths are played one-by-one. The user can move simulated objects which are part of the world around, along with loading new objects to place around the world. Care is taken to avoid these objects getting in the way of the paths. The output of this stage is a patch containing the `diff` between the original world and the modified world. A manual step is required to add the “dynamic” tag to the moved/modified objects.
Sensor Generation
The final step is to generate the simulated sensor readings which will be fed to the set of SLAM systems under test. For each combination of path and dynamics file, a stereo image stream, segmentation image stream, IMU stream and ground truth position stream are recorded to the world’s directory.
Combination of pre-generated environments and custom generated environments using free assets
Objects which are moved in future runs are marked “dynamic”, allowing them to be identified by the simulated segmentation camera
Multiple runs though the same environments with different levels of change from the baseline for different difficulties. Expect to see higher levels of inaccuracy when environment changes significantly between a run and its map generation
Camera and IMU are attached to a simulated robot capable of 5dof motion in 3d environments.
Evaluation of base performance
Evaluation of an “ideal” probability model
The best possible performance of the model can be simulated by eliminating all keypoints which are moved between an initial environment and a modified environment. This is achieved using a segmentation camera as part of the dataset. Determining which keypoints fall on dynamic objects during the initial map generation allows those keypoints to be culled prior to loading the map for subsequent runs. This provides a simulacrum of an idealized “perfect” result of the probability model by using foreknowledge of which keypoints would ideally be marked as out-of-date.
Limitations
This model for “perfect” performance is capable of identifying objects in the original dataset which are moved or deleted, but is not capable of identifying objects which become occluded by new or existing objects. The decision could be made that objects which become occluded should be marked “dynamic”, but this would raise issues if the occluded object remains visible from some perspectives contained in one or more test paths. It could be decided that occluded objects are marked “dynamic” if they are not seen due to the occlusion in a subsequent test path, but this would require additional decisions to be made (e.g. What should be done about objects which would not be seen by a path even if no occlusion exists? To what extent does an object need to be seen in order to avoid the “dynamic” mark?) For these reasons, objects may become occluded by other objects in a scene, but will not be culled by the ideal model.
Test Plan

Implementation of a probability model for the elimination of outdated map points
Definitions
The following functions and parameters are being defined here for use in the several probability model implementations discussed below.

% $$
% \text{map_{init}}
% \textbf{expectedkps}(x_t) = \left{kp\in \text{map_{init}} | kp \in \textbf{frustum}(x_t)\right}
% $$

Implementation 1
The goal of this probability model is to estimate the probability that a keypoint $$kp_n\in map_{initial}$$ exists at time $$t$$, given the current sensor input $$Z_t$$ and state estimate $$x_t$$. Bayes’ theorem is used to determine a new probability for the existence of $$kp_n$$, written as:

% $$
% p(\mathbf{exists}(kp_n, t)|Z_t,x_t)=\begin{cases}p(\mathbf{exists}(kp_n, t-1)|Z_{t-1},x_{t-1}) &: kp_n \notin \mathbf{expectedkps}(x_t)
% 1 &: kp_n\in Z_t
% \frac{p(Z_t,x_t|\mathbf{exists}(kp_n, t)) \times p(\mathbf{exists}(kp_n, t-1))}{p(Z_{t-1},x_{t-1})} &: \text{default}
% \end{cases}
% $$

This equation makes several assumptions:
If $$kp_n$$ is observed by $$Z_t$$, then we can be sure it exists.
The probability of a keypoint’s existence will not change without some observation of the area where $$kp_n$$ is expected to exist. This implies that if we would not expect $$Z_t$$ to observe $$kp_n$$ (based on $$kp_n$$’s last observed location and the current $$x_t$$), $$p(\mathbf{exists}(kp_n,t))$$ will not be changed. This means we can fix $$p(\mathbf{exists}(kp_n,t))=p(\mathbf{exists}(kp_n,t-1))$$.
Each of these assumptions will be interrogated and analyzed to determine their validity in simulated and real world examples.
Implementation 2
The assumption that the probability of a keypoint’s existence does not change as a function of time may be false. This implementation makes use of the time between the loaded map’s generation and the current time to increase the probability that a keypoint no longer exists based on the time delta to when it was last observed.
Evaluation of probability model on simulated and real-world datasets

<!-- This is going to contain a section on comparisons with other methods of dynamics removal.
This is difficult to do in practice due to the complications of using other people's numbers
in direct comparison to my own. Add a section on potential research into context independent
slam benchmarking. -->